\documentclass{beamer}
\usepackage[utf8]{inputenc}
\usepackage[T1]{fontenc}

\setbeamerfont{title}{size=\LARGE}

\title{Bash II}
\author{Eliška Jégrová}
\date{13. 10. 2025}

% pro obrázky a kreslení
\usepackage{tikz}
\usetikzlibrary{positioning}
\usepackage{graphicx}
\usepackage{datetime}
\usepackage{svg}
\usepackage{listings}


% vlastní příkaz pro tildy
\newcommand{\ts}{\raisebox{-0.25em}{\textasciitilde}}

\begin{document}
	
	\frame{\titlepage}
	
	\begin{frame}{Obsah}
		\tableofcontents
	\end{frame}
	
\section{Roury a filtry}

\begin{frame}[fragile]{Roury a filtry}
	\begin{itemize}
		\item Unixový shell umí spojovat příkazy dohromady.
		\item Každý příkaz má standardní vstup a výstup.
		\item Např. jak bychom zjistili pomocí počítače, který soubor má nejvíce řádků a který nejméně?
	\end{itemize}
	\vspace{0.5em}
	\hspace{4.85em}\$ cd \texttt{\ts}/Dokumenty/data-shell/molecules
	\begin{lstlisting}[language=bash]
	$ wc -l *.pdb
	 20 cubane.pdb
	 12 ethane.pdb
  	 9 methane.pdb
	 30 octane.pdb
	 21 pentane.pdb
	 15 propane.pdb
	 107 celkem
	\end{lstlisting}
		* - odpovídá libovolnému počtu znaků
\end{frame}

\begin{frame}[fragile]{Přesměrování výstupu}
	\begin{itemize}
		\item Výstup se standardně vypisuje do terminálu.
	\end{itemize}
	\begin{center}
		\includesvg[width=0.5\textwidth]{wc-to-term.svg}
	\end{center}
	\begin{itemize}
		\item Pomocí \texttt{>} lze přesměrovat výstup příkazu do souboru.
	\end{itemize}
	    \begin{center}
		\includesvg[width=0.5\textwidth]{wc-to-file.svg}
	\end{center}
	\begin{lstlisting}[language=bash]
	$ wc -l *.pdb > delky.txt
	$ cat delky.txt
	\end{lstlisting}
	Pozor: pokud soubor existuje, jeho obsah se přepíše.
\end{frame}

\begin{frame}[fragile]{Hledani nejmensi molekuly}
	\begin{itemize}
	\item Řazení: sort
		\begin{itemize}
			\item Přepínač \texttt{-n} zapíná číselné řazení.
		\end{itemize}
	\begin{lstlisting}[language=bash]
	$ sort delky.txt
	$ sort -n delky.txt
	\end{lstlisting}

\item Jak najít nejmenší molekulu?
		\end{itemize}
	\begin{lstlisting}[language=bash]
   $ wc -l *.pdb > delky.txt
   $ sort -n delky.txt > delky-serazene.txt
   $ head -n1 delky-serazene.txt
   9 methane.pdb
	\end{lstlisting}
\end{frame}

\begin{frame}[fragile]{Přesměrování vstupu}
	\$ cat
	\begin{center}
	\includesvg[width=0.4\textwidth]{cat.svg}
	\end{center}	

	\$ cat < delky.txt	
	\begin{center}
	\includesvg[width=0.4\textwidth]{cat-from-file.svg}
	\end{center}	
	
	\$ cat delky.txt	
	\begin{center}
		\includesvg[width=0.6\textwidth]{cat-file-arg.svg}
	\end{center}	
\end{frame}

\begin{frame}[fragile]{Roury (pipes)}
	
	\hspace{1em} \$ sort -n delky.txt > delky-serazene.txt	 \\
	\hspace{1em} \$ head -n1 < delky-serazene.txt
		\begin{center}
		\includesvg[width=0.8\textwidth]{sort-head-nopipe.svg}
	\end{center}
	\begin{itemize}
		\item Pomocí \texttt{|} přesměruj výstup jednoho příkazu na vstup druhého.
	\end{itemize}
	\hspace{2em} \$ sort -n delky.txt | head -n1
	\begin{center}
		\includesvg[width=0.7\textwidth]{sort-head-pipeline.svg}
	\end{center}

	\hspace{2em} \$ wc -l *.pdb | sort -n | head -n 1
	\begin{center}
	\includesvg[width=0.7\textwidth]{pipeline.svg}
	\end{center}	
\end{frame}

\begin{frame}{Samostatná práce – Analýza molekul}
	\begin{itemize}	
		\item Najdi nejdelší molekulu ve složce \texttt{\ts}/Dokumenty/data-shell/molecules
	\end{itemize}
	\vspace{1.2em}
	\begin{enumerate}
	\item Pomocí přesměrování do souboru	
		\begin{itemize}	
		\item Spočítej počet řádků všech \texttt{.pdb} souborů a ulož do \texttt{delky.txt}.  
		\item	Seřaď soubor \texttt{delky.txt} číselně a ulož do \texttt{delky-serazene.txt}. 	 
		\item Vypiš poslední řádek souboru – to je nejdelší molekula.
		\end{itemize}	
	\vspace{0.5em}
	\item Pomocí roury (pipe)
		\begin{itemize}		 
		\item Spočítej počet řádků a hned seřaď číselně a vypiš poslední řádek (jednořádkový příkaz)
		\end{itemize}	
	\end{enumerate}	

	 \begin{itemize}
	 \item Najdi tři nejkratší molekuly (bonus)
	 \end{itemize}
	\hspace{1.8em} -- vyber si způsob jak úkol udělat

\end{frame}

\section{Echo}

\begin{frame}[fragile]{Příkaz \texttt{echo}}
	\begin{itemize}
		\item Slouží k výpisu textu do terminálu.
	\end{itemize}
	
	\hspace{3em}\texttt{\$ echo Ahoj světe!} \\
	\hspace{3em}Ahoj světe! \\[0.5em]
	
	\hspace{3em}\texttt{\$ echo Ahoj\ \ \ \ \ světe!} \\
	\hspace{3em}Ahoj    světe! \\[0.5em]
	
	\hspace{3em}\texttt{\$ echo \textquotesingle Ahoj\ \ \ \ \ \ \ \ \ světe!\textquotesingle} \\
	\hspace{3em}Ahoj\ \ \ \ \ \ \ \ \ světe!

	\begin{itemize}
		\item Jednoduché vs Dvojité uvozovky
	\end{itemize}

	\hspace{3em}\$ echo \$0 \\
	\hspace{3em}bash \\[0.5em]
	
	\hspace{3em}\$ echo "\$0" \\
	\hspace{3em}bash \\[0.5em]
	
	\hspace{3em}\$ echo \textquotesingle \$0\textquotesingle \\
	\hspace{3em}\$0 \\[0.5em]
\end{frame}

\begin{frame}[fragile]{Příkaz \texttt{echo}}
	\begin{itemize}
		\item Echo do souboru - přepíše se soubor
	\end{itemize}
	\hspace{3em} \$	echo haló haló > basnicka.txt \\
	\hspace{3em} \$ cat basnicka.txt\\
	\hspace{3em}	haló haló\\

	\vspace{1em}
	\begin{itemize}
		\item Echo do souboru - přidat na konec souboru
	\end{itemize}
	\hspace{3em} \$	echo haló haló >{}> basnicka.txt \\
	\hspace{3em} \$ cat basnicka.txt\\
	\hspace{3em}	haló haló\\
	\hspace{3em}    co se stalo?
\end{frame}

\begin{frame}{Samostatná práce – Echo a soubory}
	\begin{enumerate}
		\item Vytvoř soubor \texttt{bajka.txt} a napiš do něj první řádek textu „Moje první báseň“ pomocí \texttt{echo} a přesměrování \texttt{>}.
		\item Přidej další dva řádky s textem „Druhý řádek“ a „Třetí řádek“ na konec souboru pomocí \texttt{>{}>}.
		\item Zobraz celý obsah souboru a ověř, že obsahuje tři řádky.
		\item Bonus: Přidej jeden řádek, kde budeš mít víc mezer mezi slovy, a ověř, že se mezery zachovají při výpisu.
	\end{enumerate}
\end{frame}

\section{Hledání souborů a textu}

\begin{frame}[fragile]{Příkaz \texttt{grep}}
	\begin{itemize}
		\item Slouží k hledání řetězce v souborech.
		\item Základní syntaxe: \texttt{grep <výraz> <soubor>}
	\end{itemize}
	\hspace{2em}\$ cd \texttt{\ts}/Dokumenty/data-shell/writing \\
	\hspace{2em}\texttt{\$ grep se haiku.txt} \\
	\hspace{2em}– vypíše všechny řádky obsahující výraz "se"
	\vspace{0.5em}
	\begin{itemize}
		\item Přepínače:
		\begin{itemize}
			\item \texttt{-i} – ignoruje velikost písmen
			\item \texttt{-n} – zobrazí čísla řádků
			\item \texttt{-w} – hledá jen celá slova
		\end{itemize}
	\end{itemize}
\end{frame}

\begin{frame}[fragile]{Příkaz \texttt{find}}
	\begin{itemize}
		\item Slouží k hledání souborů v adresářové struktuře.
		\item Základní syntaxe: \texttt{find <cesta> <kritéria>}
	\end{itemize}
	
	\hspace{2em}\texttt{\$ find . -name \textquotesingle *.txt\textquotesingle} \\
	\hspace{2em}– najde tyto soubory v aktuálním adresáři a podsložkách
	\vspace{0.5em}
	\begin{itemize}
	\item Stejný výsledek se dá získat pomocí grepu
	\end{itemize}
	\hspace{2em}\texttt{\$ ls --recursive | grep txt} \\
\end{frame}

\begin{frame}{Samostatná práce – Hledání textu a souborů}
	\begin{itemize}
		\item V adresáři \texttt{\ts}/Dokumenty/data-shell/writing najděte všechny soubory \texttt{.md}.
		\begin{itemize}
			\item Použijte příkaz \texttt{find}
		\end{itemize}
	\end{itemize}
		\vspace{0.5em}
	Pro následující úkoly použijte soubor data-shell/writing/data/LittleWomen.txt
			\vspace{0.2em}
		\begin{itemize}
		\item V souboru najděte všechny řádky obsahující slovo \texttt{Unlike}.
		\begin{itemize}
			\item Použijte \texttt{grep}.
			\item Vyzkoušejte přepínač \texttt{-i} pro ignorování velikosti písmen.
		\end{itemize}
		\vspace{0.5em}
		\item Vypište čísla řádků, kde se vyskytuje slovo \texttt{reason}.
		\begin{itemize}
			\item Použijte \texttt{grep -n}.
		\end{itemize}
		\vspace{0.5em}
		\item Najděte všechny řádky obsahující celá slova \texttt{echo}.
		\begin{itemize}
			\item Použijte přepínač \texttt{-w}.
		\end{itemize}
	\end{itemize}
\end{frame}



\end{document}
