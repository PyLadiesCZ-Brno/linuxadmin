\documentclass{beamer}
\usepackage[utf8]{inputenc}
\usepackage[T1]{fontenc}
\usepackage[czech]{babel}

\setbeamerfont{title}{size=\LARGE}

\title{Instalace balíčků}
\author{Eliška Jégrová}
\date{11. 11. 2025}

% pro obrázky a kreslení
\usepackage{tikz}
\usetikzlibrary{positioning}
\usepackage{graphicx}
\usepackage{datetime}
\usepackage{svg}
\usepackage{listings}
\usepackage{dirtree}


% vlastní příkaz pro tildy
\newcommand{\ts}{\raisebox{-0.25em}{\textasciitilde}}

\begin{document}
	
	\frame{\titlepage}
	
	\begin{frame}{Obsah}
		\tableofcontents
	\end{frame}

\section{Balíčky v Linuxu}

\begin{frame}{Balíčky v Linuxu}
	\begin{itemize}
		\item Programy v Linuxu jsou většinou baleny do \textbf{balíčků}.
		\item Balíčky obsahují:
		\begin{itemize}
			\item vlastní program nebo knihovny,
			\item informace o verzi,
			\item seznam závislostí (co je nutné mít nainstalováno),
			\item konfigurace,
			\item manuálové stránky, dokumentace,
			\item instalační skripty (ne nutně).
		\end{itemize}
		\item Každá distribuce má svůj vlastní systém správy balíčků.
	\end{itemize}
\end{frame}



\begin{frame}{Základní a nadstavbové systémy}
	\begin{itemize}
		\item Na systémech jsou obvykle dva balíčkovací systémy:
		\begin{itemize}
			\item \textbf{Základní:} stará se o instalaci a evidenci (\texttt{rpm})
			\item \textbf{Nadstavbová:} řeší stahování a závislosti (\texttt{dnf})
		\end{itemize}
		
		\vspace{1em}
		\begin{itemize}
			\item Všechny spravují software dostupný všem uživatelům
			\item Pro instalaci je potřeba oprávnění superuživatele
		\end{itemize}
		
	\end{itemize}
\end{frame}

\begin{frame}{Různé distribuce, různé nástroje}
	\begin{center}
		\begin{tabular}{|l|l|}
			\hline
			\textbf{Distribuce} & \textbf{Základní systém správy balíčků} \\
			\hline
			Debian, Ubuntu & APT (dpkg) \\
			Fedora, CentOS & DNF (RPM) \\
			openSUSE & Zypper (RPM) \\
			Arch Linux & Pacman \\
			\hline
		\end{tabular}
	\end{center}
\end{frame}


\section{RPM}
\begin{frame}[fragile]{RPM – informace o balíčcích}
	\begin{itemize}
		\item Dotaz na verzi balíčku

        \texttt{\$ rpm -q bash \\
		bash-5.2.37-1.fc42.x86\_64 \\
		\$ rpm -q kernel \\
		kernel-6.17.7-200.fc42.x86\_64}
	
		\vspace{0.5em}
		
		\texttt{\$ rpm -qi bash \\
			Name        : bash \\
			Version     : 5.2.37 \\
			Release     : 1.fc42 \\
			Architecture: x86\_64 \\
			Install Date: Wed 09 Apr 2025 02:05:13 PM CEST \\
			Group       : Unspecified \\
			Size        : 8547020 \\
			License     : GPL-3.0-or-later \\
			...}
		\end{itemize}
\end{frame}

\begin{frame}[fragile]{RPM – informace o balíčcích}
	\begin{itemize}
		\item Zobraz všechny soubory které balíček obsahuje
				
		\texttt{\$ rpm -q -l bash \\
			/etc/skel/.bash\_logout \\
			/etc/skel/.bash\_profile \\
			/etc/skel/.bashrc \\
			/usr/bin/alias \\
			/usr/bin/bash \\
			/usr/bin/bashbug \\
			/usr/bin/bashbug-64 \\
			/usr/bin/bg \\
			/usr/bin/cd \\
			/usr/bin/command \\
			/usr/bin/fc \\
			...}

	\end{itemize}
\end{frame}

\begin{frame}[fragile]{RPM – závislosti}
	\begin{itemize}
		\item Dotaz na závislosti balíčku (co potřebuje bash)	\\		
			\texttt{\$ rpm -q -{}-requires bash\\
			/usr/bin/sh\\
			config(bash) = 5.2.37-1.fc42\\
			filesystem >= 3\\
			libc.so.6()(64bit)\\
			libc.so.6(GLIBC\_2.11)(64bit)}

		\vspace{1em}
		\item Dotaz - jaký balíček poskytuje danou závislost
			
			\texttt{\$ rpm -q -{}-whatprovides 'libc.so.6()(64bit)'}
			\texttt{glibc-2.41-11.fc42.x86\_64}

	\end{itemize}
\end{frame}

	\begin{frame}{Samostatné práce 1 – Informace o balíčku}
	Zjisti informace o balíčcích \texttt{coreutils} a \texttt{firefox}:
	\begin{itemize}
		\item Verze balíčků
		\item Podrobné informace (-i)
		\item Seznam souborů, které balíček obsahuje (-l)
	\end{itemize}
\end{frame}

\begin{frame}{Samostatné práce 2 – Závislosti balíčku}
	Zjisti závislosti a poskytovatele knihoven:
	\begin{itemize}
		\item Zjisti, jak se jmenuje balíček pro vim (/usr/bin/vim)
		\item Jaké závislosti má balíček \texttt{vim}? (requires)\\
		\item Který balíček poskytuje \texttt{libacl.so.1()(64bit)}? (whatprovides)\\
	\end{itemize}
\end{frame}


\section{Repozitář}
\begin{frame}{Repozitář}
	\begin{itemize}
		\item Příkaz \texttt{rpm} nekomunikuje po internetu – neumí vyhledávat balíčky, které nejsou nainstalované
		\vspace{0.8em}

		\item Nemůžeme zjistit, který balíček obsahuje určitý soubor, pro nenainstalované programy:

			\texttt{\$ rpm -q -{}-whatprovides /usr/bin/sl}\\
			chyba: soubor \texttt{/usr/bin/sl}: Adresář nebo soubor neexistuje

		\item K tomu je potřeba prohlédnout repozitář – seznam všeho softwaru, který Fedora poskytuje
	\end{itemize}
\end{frame}

\section{DNF}
\begin{frame}{Repozitář – práce s \texttt{dnf}}
	\begin{itemize}
		\item Obdoba příkazu \texttt{rpm -q} je \texttt{dnf repoquery}:

	 	\texttt{\$ dnf repoquery -{}-whatprovides /usr/bin/sl}\\[0.2em]
		\texttt{Fedora 42 - x86\_64 \hspace{1em} 6.8 MB/s | 74 MB 00:10}\\
		\texttt{sl-0:5.02-23.fc42.x86\_64}

		\vspace{1em}
		\item Při prvním použití stahuje metadata o repozitářích.
		\item Při dalším spuštění zobrazí jak stará metadata má k dispozici
	\end{itemize}
\end{frame}

\begin{frame}{Instalace a odinstalace}
	\begin{itemize}
		\item \texttt{rpm} umí instalovat (\texttt{-i}) a mazat (\texttt{-e}) balíčky, ale:
		\begin{itemize}
			\item neřeší závislosti ani stahování,
			\item hodí se jen ve zvláštních případech (např. odpojené systémy).
		\end{itemize}
		\vspace{1em}
		\item Pro běžnou správu se používá \texttt{dnf}:
		
			 \texttt{\$ dnf install <balíček>} \\
			 \texttt{\$ dnf remove <balíček>} \\
			 \texttt{\$ dnf update}
		

		\vspace{1em}
		\item Příklad:  \\
		\texttt{\$ sudo dnf install sl}\\
		\texttt{\$ sudo dnf remove -y htop}
		\vspace{1em}
		\item Přepínač \texttt{-y} = automaticky odpovídá „ano“
	\end{itemize}
\end{frame}

\section{Konfigurace repozitáře}
\begin{frame}{Repozitáře}
	\begin{itemize}
		\item konfigurace repozitářů v \texttt{/etc/yum.repos.d/}
		\item Fedora neobsahuje některé balíčky (např. MP3, videa) kvůli patentům.
\item RPM Fusion = evropský projekt s volným i uzavřeným softwarem:
\begin{itemize}
	\item \texttt{free} – open source, volně používat, měnit, zkoumat
	\item \texttt{nonfree} – uzavřený software, např. Steam
\end{itemize}
\vspace{1em}
\item zkus nainstalovat program \texttt{mplayer}
	\end{itemize}
\end{frame}

\begin{frame}{Instalace repozitáře RPM Fusion}
	\begin{itemize}
		\item https://rpmfusion.org/
	\end{itemize}
		\begin{itemize}
		\item Při první instalaci se DNF zeptá, zda důvěřuje GPG klíčům.
		\begin{itemize}
			\item lze ověřit na rpmfusion.org/keys
		\end{itemize}
		\vspace{0.5em}
		\item Po úspěšné instalaci bude balíček \texttt{rpmfusion-free-release} obsahovat konfiguraci repozitáře. \\
		\texttt{\$ rpm -ql rpmfusion-free-release
			/etc/pki/rpm-gpg/RPM-GPG-KEY-rpmfusion-free-fedora-2020
			/etc/pki/rpm-gpg/RPM-GPG-KEY-rpmfusion-free-fedora-latest
			/etc/yum.repos.d/rpmfusion-free-updates.repo
			/etc/yum.repos.d/rpmfusion-free.repo \\
			...}
	\end{itemize}
\end{frame}

	\begin{frame}{Samostatné práce 3 – Práce s repozitáři a dnf}
		Zjisti balíček, který obsahuje program \texttt{terminator}, i když není nainstalovaný (/usr/bin/terminator)
		\begin{itemize}
			\item Nainstaluj balíček
			\item Odinstaluj balíček
		\end{itemize}
	\end{frame}

	\begin{frame}{Samostatné práce 4 – Instalace specifické verze balíčku }
	Zjisti verze balíčku htop, které se dají nainstalovat (repoquery)
	\begin{itemize}
		\item Nainstaluj starší verzi: 3.3.0
		\item Nainstaluj nejnovější verzi.
	\end{itemize}
\end{frame}

\end{document}