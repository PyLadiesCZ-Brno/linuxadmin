\documentclass{beamer}
\usepackage[utf8]{inputenc}
\usepackage[T1]{fontenc}

\setbeamerfont{title}{size=\LARGE}

\title{Bash III}
\author{Eliška Jégrová}
\date{2. 11. 2025}

% pro obrázky a kreslení
\usepackage{tikz}
\usetikzlibrary{positioning}
\usepackage{graphicx}
\usepackage{datetime}
\usepackage{svg}
\usepackage{listings}
\usepackage{dirtree}


% vlastní příkaz pro tildy
\newcommand{\ts}{\raisebox{-0.25em}{\textasciitilde}}

\begin{document}
	
	\frame{\titlepage}
	
	\begin{frame}{Obsah}
		\tableofcontents
	\end{frame}
	
	\section{Proměnné}
	
	\begin{frame}[fragile]{Proměnné}
		\begin{itemize}
			\item Příklad přiřazení
			\begin{lstlisting}[language=bash]
	$ jmeno=minotaur.dat
			\end{lstlisting}
			
			\item Obsah proměnné
			\begin{lstlisting}[language=bash]
	$ echo $jmeno
			\end{lstlisting}
			
			\item Proměnnou lze použít v příkazu, např.:
			\begin{lstlisting}[language=bash]
	$ head -n2 $jmeno | tail -n1
			\end{lstlisting}
			
			\item Hodnota proměnné můžeme přepsat
			\begin{lstlisting}[language=bash]
	$ jmeno=unicorn.dat
	$ head -n2 $jmeno | tail -n1
			\end{lstlisting}
			
		\end{itemize}
	\end{frame}
	
	\section{Cyklus for}
	
	\begin{frame}[fragile]{Cyklus \texttt{for}}
		\begin{itemize}
			
			\item Syntaxe:
			\begin{lstlisting}[language=bash]
  for prvek in seznam
    do
    prikaz
  done
			\end{lstlisting}
			
			\item Příklad s proměnnou:
			\begin{lstlisting}[language=bash]
  for jmeno in minotaur.dat unicorn.dat
    do
    head -n1 $jmeno
  done
			\end{lstlisting}
			
			\item Příklad - vypsání souborů:
			\begin{lstlisting}[language=bash]
  for soubor in *.dat; do
    echo $soubor
  done
			\end{lstlisting}
			
		\end{itemize}
	\end{frame}
	
	\begin{frame}[fragile]{Cyklus \texttt{for} – vnořené cykly}
		\begin{itemize}
			\item Příklad: Vytvořit složky z názvů molekul a určitých teplot        
			\begin{lstlisting}[language=bash]
  for molekula in cubane ethane methane
  do
      for teplota in 25 30 37 40
      do
          echo mkdir $molekula-$teplota
      done
  done
			\end{lstlisting}
		\end{itemize}
	\end{frame}

	\begin{frame}[fragile]{Samostatná práce 1 – Proměnné a cykly}
		\begin{itemize}
			\item \textbf{Cíl:} procvičit práci s proměnnými a cyklem \texttt{for} při tvorbě více složek.
			\item Vytvoř proměnnou \texttt{projekty} s názvy projektů "web api data"
			\item Pomocí cyklu vytvoř adresáře s názvy z proměnné.
			\item Zkontroluj vytvoření pomocí \texttt{ls}.
			\item Bonus: Pomocí cyklu vytvoř v každém projektu podadresář \texttt{logs}, aby struktura vypadala takto:
		\end{itemize}

			\dirtree{%
				.1 ..
				.2 api.
				.3 logs.
				.2 data.
				.3 logs.
				.2 web.
				.3 logs.
			}

		\begin{itemize}
			\item Bonus 2: přidej do každého podadresáře soubor \texttt{info.txt} \\s textem "Projekt <jméno>".
		\end{itemize}
	\end{frame}

	\section{Proměnné prostředí}
	
	\begin{frame}[fragile]{Proměnné prostředí}
		\begin{itemize}
			\item Proměnné bashe - jsou dostupné jen pro samotný Bash
			\item Proměnné prostředí - dostupné i pro běžící programy 
			\vspace{2em}
			\item Příkaz \texttt{env} vypíše všechny proměnné prostředí:
			\begin{lstlisting}[language=bash]
$ env | head
$ env | grep ^LANG
			\end{lstlisting}
		\end{itemize}
	\end{frame}
	
	\begin{frame}[fragile]{Exportování}
		\begin{itemize}
			\item způsob, jak předat proměnnou programům
			\begin{lstlisting}[language=bash]
$ export jmeno=basilisk.dat
$ export jmeno
			\end{lstlisting}
			
			\item Příklad
		\end{itemize}
		
		\begin{columns}[T]
			\begin{column}{0.48\textwidth}
				\textbf{Bez exportu:}
				\begin{lstlisting}[language=bash]
  jmeno=minotaur.dat
  env | grep jmeno
  echo $jmeno
  minotaur.dat
				\end{lstlisting}
			\end{column}
			\begin{column}{0.48\textwidth}
				\textbf{S exportem:}
				\begin{lstlisting}[language=bash]
  export jmeno
  env | grep jmeno
  jmeno=minotaur.dat
  echo $jmeno
  minotaur.dat
				\end{lstlisting}
			\end{column}
		\end{columns}
	\end{frame}
	
	
	\begin{frame}[fragile]{Složení názvů proměnných, Výzva}
		\begin{itemize}
			\item Pokud chceš mít text za proměnnou (např. „ovoce“ v „jablko“), použij složené závorky:
			\begin{lstlisting}[language=bash]
$ majitel=Petr
$ echo ${majitel}ovo jablko    
			\end{lstlisting}
			\vspace{2em}
			\item Bash má vlastní proměnné používané pro prompt: \texttt{PS1}, \texttt{PS2}.  
			\begin{itemize}
				\item Příklad přenastavení výzvy:
				\begin{lstlisting}[language=bash]
$ PS1='[0_0]'
				\end{lstlisting}
			\end{itemize}
		\end{itemize}
	\end{frame}
	
	\begin{frame}[fragile]{Samostatná práce 2 – Proměnné prostředí a výzva}
		\begin{itemize}
			\item \textbf{Cíl:} vyzkoušet práci s proměnnými prostředí a úpravou výzvy.
			\item Nastav proměnnou \texttt{jmeno} s tvým jménem  a exportuj ji.
			\item Ověř, že je dostupná v \texttt{env}
			\item Uprav výzvu (\texttt{PS1}), aby ukazovala tvoje jméno
			\item Bonus: Přidej k výzvě i aktuální adresář \texttt{\textbackslash w}.
		\end{itemize}
	\end{frame}

	\section{Skripty}
	
	\begin{frame}[fragile]{Skripty}
		\begin{itemize}
			\item Skripty jsou jednoduché programy – často spojují existující příkazy.
			\item Ukázka obsahu \texttt{klasifikace}:
			\begin{lstlisting}[language=bash]
  for x in *.dat
  do
    head -n2 $x | tail -n1
  done
			\end{lstlisting}
		\end{itemize}
	\end{frame}
	
	\begin{frame}[fragile]{Spustitelný skript}
		\begin{itemize}
			\item Skript se spouští pomocí Bashe:
			\begin{lstlisting}[language=bash]
$ bash klasifikace
			\end{lstlisting}
			\vspace{1em}
			\item Aby šel spustit jako příkaz, je třeba:
			\begin{enumerate}
				\item Přidat \textbf{shebang} na první řádek:
				\begin{lstlisting}[language=bash]
#! /bin/bash
				\end{lstlisting}
				\item Nastavit spustitelný příznak:
				\begin{lstlisting}[language=bash]
$ chmod +x klasifikace
				\end{lstlisting}
				\item Spustit:
				\begin{lstlisting}[language=bash]
$ ./klasifikace
				\end{lstlisting}
			\end{enumerate}
		\end{itemize}
	\end{frame}
	
	
	\begin{frame}[fragile]{Argumenty skriptu}
		\begin{itemize}
			\begin{lstlisting}[language=bash]
$ ./klasifikuj minotaur.dat
			\end{lstlisting}
			\item Argumenty z příkazové řádky se předávají pomocí speciálních proměnných
			\begin{itemize}
				\item \texttt{"\$0"} – jméno skriptu
				\item \texttt{"\$1"} – první argument
				\item  \texttt{"\$2"}, \texttt{"\$3"} – další argumenty
				\item \texttt{"\$@"} – všechny argumenty
			\end{itemize}
			\vspace{1em}
			\item Použití ve skriptu:
			\begin{lstlisting}[language=bash]
#! /bin/bash
head -n2 "$1" | tail -n1
			\end{lstlisting}
			
		\end{itemize}
	\end{frame}
	
	\begin{frame}[fragile]{Proměnná \texttt{PATH}}
		\begin{itemize}
			\item Systém hledá spustitelné programy v adresářích uvedených v proměnné \texttt{PATH}.
			\begin{lstlisting}
$ echo $PATH
/usr/local/bin:/usr/bin:/bin:/usr/local/sbin
			\end{lstlisting}
			\item Skript přesuneme do složky v \texttt{PATH}, např. \texttt{\$HOME/bin}
			\item Spustíme skript jen názvem:
			\begin{lstlisting}[language=bash]
$ klasifikace *.dat
			\end{lstlisting}
			\item K trvalé změně \texttt{PATH} je potřeba upravit ji v souboru \texttt{.bashrc}
		\end{itemize}
	\end{frame}
	
	
	
	\section{Kombinace příkazů}
	
	\begin{frame}[fragile]{Další kombinace příkazů}
		\begin{itemize}
			\item Příkazy lze řetězit pomocí \texttt{;} – provedou se postupně:
			\begin{lstlisting}
$ echo "Seznam souboru:"; ls
			\end{lstlisting}
			\vspace{1em}
			\item Operátor \texttt{\&\&} – druhý příkaz se provede jen pokud první uspěl:
			\begin{lstlisting}
$ mkdir test && echo "Slozka vytvorena."
			\end{lstlisting}
			\vspace{1em}        
			\item Operátor \texttt{||} – druhý příkaz se provede jen pokud první selhal:
		\end{itemize}

	\hspace{1.7em}	\texttt{\$ cat neexistujici.txt || echo "Soubor nenalezen."}
	\end{frame}
	
\begin{frame}[fragile]{Samostatná práce 3 – Vytvoření skriptu}
	\begin{itemize}
		\item \textbf{Cíl:} vytvořit jednoduchý skript, který vypíše základní informace o systému.
		\item Vytvoř soubor \texttt{info} s obsahem:
		\begin{lstlisting}
	echo "Uzivatel: $USER"
	echo "Aktualni adresar: $(pwd)"
	echo "Datum: $(date)"
		\end{lstlisting}
		\item Nastav skript jako spustitelný
		\item Přesuň ho do adresáře \texttt{\$HOME/bin} (vytvoř jej, pokud neexistuje).
		\item Spusť ho odkudkoliv jen názvem:
		\begin{lstlisting}[language=bash]
	$ info
		\end{lstlisting}
	\end{itemize}
\end{frame}

\section{Programy, funkce a aliasy}
	
	\begin{frame}[fragile]{Jak Bash hledá příkazy}
		\begin{itemize}
			\item Příkaz \texttt{type} ukáže, k jakému typu příkaz patří
			\item Když do Bashe zadáš jméno příkazu (např. \texttt{ls}), Bash hledá v několika druzích příkazů:
			\begin{itemize}
				\item programy v adresářích z \texttt{\$PATH}
				\begin{lstlisting}[language=bash]
  $ type cat
  cat je /usr/bin/cat
				\end{lstlisting}
				\item zabudované příkazy (built-in)
				\begin{lstlisting}[language=bash]
  $ type cd
  cd je soucast shellu
				\end{lstlisting}
				\item aliasy – zkratky pro jiné příkazy
				\begin{lstlisting}
  $ type ll
  ll je alias na "ls -l --color=auto"

				\end{lstlisting}
				\item funkce shellu – skripty přímo definované v Bashi
				\begin{lstlisting}[language=bash]
  $ type quote
  quote je funkce
				\end{lstlisting}
			\end{itemize}
		\end{itemize}
	\end{frame}
	
	\begin{frame}[fragile]{Skript pro vytvoření a vstup do složky}
		\begin{itemize}
			\item Cílem - skript, který vytvoří novou složku a přepne se do ní.
			\item Obsah skriptu \texttt{mcd}:
			\begin{lstlisting}[language=bash]
  mkdir -p $1
  cd $1
			\end{lstlisting}
			
			\item Spuštění:
			\begin{lstlisting}[language=bash]
  $ bash mcd novy-adresar
			\end{lstlisting}
			\item Nový proces spouští skript – změna adresáře se ale týká jen tohoto podprocesu.
		\end{itemize}

		\centering
		\includesvg[width=0.7\textwidth]{forking-mcd.svg}
	\end{frame}
	
	\begin{frame}[fragile]{Zabudované příkazy a \texttt{source}}
		\begin{itemize}
			\item Některé příkazy jsou součástí samotného shellu – tzv. \textbf{zabudované (built-in)} příkazy.
			\begin{itemize}
				\item Např. \texttt{cd}, \texttt{export}, \texttt{alias}, \texttt{source}
				\item Nespouštějí nový proces – mění přímo prostředí aktuálního shellu
			\end{itemize}
			
			\vspace{0.5em}
			\item Ověření, zda je příkaz zabudovaný:
			\begin{lstlisting}[language=bash]
	type cd
	# cd is a shell builtin
			\end{lstlisting}
			
			\vspace{0.5em}
			\item Příkaz \texttt{source} (nebo zkráceně \texttt{.}) spustí skript v aktuálním shellu.
			\begin{lstlisting}[language=bash]
	source mcd novy-adresar
			\end{lstlisting}
			\item Díky tomu zůstaneš v nové složce i po skončení skriptu –
			na rozdíl od běžného spuštění, které vytváří nový proces.
		\end{itemize}
	\end{frame}
	
	
	\begin{frame}[fragile]{Alias – jednoduchá zkratka}
		\begin{itemize}
			\item Alias je zkratka - nahrazuje jeden příkaz jiným
			\begin{lstlisting}
	alias gs='git status'
			\end{lstlisting}
			\item Použití: \\ \hspace{1em}když napíšeš \texttt{ll /home}, Bash místo toho spustí \\ \hspace{3em}\texttt{ls -l --color=auto /home}.
			\item Zrušení aliasu:
			\begin{lstlisting}
	unalias gs
			\end{lstlisting}
		\end{itemize}
	\end{frame}
	
	\begin{frame}[fragile]{Funkce v Bashi}
		\begin{itemize}
			\item Funkce je více „robustní“ zkratka – může obsahovat více příkazů, logiku, podmínky.
			\item Definice:
			\begin{lstlisting}[language=bash]
mcd() {
	mkdir -p "$1"
	cd "$1"
	}
			\end{lstlisting}
			\item Použití:
			\begin{lstlisting}[language=bash]
	mcd projekty
			\end{lstlisting}
			\item Argumenty uvnitř: \texttt{\$1}, \texttt{\$2}, …, počet v \texttt{\$\#}, všechny v \texttt{\$@}. 
			\item Užitečný příklad – kombinace mkdir + cd:
			\begin{lstlisting}[language=bash]
	mcd() {
	 mkdir -p "$1" && cd "$1"
	}
			\end{lstlisting}
		\end{itemize}
	\end{frame}
	
	\begin{frame}[fragile]{Zapamatování aliasů \& funkcí (např. v \texttt{.bashrc})}
		\begin{itemize}
			\item Alias nebo funkci definujeme ručně, ale zmizí po ukončení Bashe.
			\item Řešení: vložit je do souboru \texttt{\ts/.bashrc}, který se načte při startu shellu:
			\begin{lstlisting}
	alias gs='git status'
	mcd() {
		mkdir -p "$1" && cd "$1"
	}
			\end{lstlisting}
			\item Po úpravě spustíme: \\
			\hspace{2em}\texttt{\$ source \ts/.bashrc}
			\item Nebo otevřít nový terminál — definice se načtou.
		\end{itemize}
	\end{frame}
	
	\begin{frame}[fragile]{Samostatná práce 4 – Typy příkazů, aliasy a funkce}
	\begin{itemize}
		\item \textbf{Cíl:} pochopit rozdíl mezi typy příkazů a vytvořit vlastní alias a funkci.
		\item Zjisti typ příkazů \texttt{ls}, \texttt{source}, \texttt{alias}
		\item Vytvoř alias pro často používaný příkaz, např. něco z gitu.
		\item Definuj jednoduchou funkci \texttt{greet}, která bude obsahovat
		\begin{lstlisting}
	echo "Ahoj $USER!"
		\end{lstlisting}
		\item Spusť:
		\begin{lstlisting}[language=bash]
	$ greet
		\end{lstlisting}
		\item Bonus: přidej alias i funkci do \texttt{\ts/.bashrc} a načti je v terminálu (source)
	\end{itemize}
	\end{frame}
	
\end{document}