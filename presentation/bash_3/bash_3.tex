\documentclass{beamer}
\usepackage[utf8]{inputenc}
\usepackage[T1]{fontenc}

\setbeamerfont{title}{size=\LARGE}

\title{Bash III}
\author{Eliška Jégrová}
\date{2. 11. 2025}

% pro obrázky a kreslení
\usepackage{tikz}
\usetikzlibrary{positioning}
\usepackage{graphicx}
\usepackage{datetime}
\usepackage{svg}
\usepackage{listings}


% vlastní příkaz pro tildy
\newcommand{\ts}{\raisebox{-0.25em}{\textasciitilde}}

\begin{document}
	
	\frame{\titlepage}
	
	\begin{frame}{Obsah}
		\tableofcontents
	\end{frame}
	
\section{Proměnné}

\begin{frame}[fragile]{Proměnné}
	\begin{itemize}
		\item Příklad přiřazení
		\begin{lstlisting}[language=bash]
	  $ jmeno=minotaur.dat
		\end{lstlisting}
	
		\item Obsah proměnné
		\begin{lstlisting}[language=bash]
  	  $ echo $jmeno
		\end{lstlisting}
		
		\item Proměnnou lze použít v příkazu, např.:
		\begin{lstlisting}[language=bash]
	  $ head -n2 $jmeno | tail -n1
		\end{lstlisting}
		
		\item Hodnota proměnné můžeme přepsat
		\begin{lstlisting}[language=bash]
	  $ jmeno=unicorn.dat
	  $ head -n2 $jmeno | tail -n1
		\end{lstlisting}
		
	\end{itemize}
\end{frame}

\section{Cyklus for}

\begin{frame}[fragile]{Cyklus \texttt{for}}
	\begin{itemize}

	\item Syntaxe:
	\begin{lstlisting}[language=bash]
  for prvek in seznam
  do
    prikaz
  done
	\end{lstlisting}

	\item Příklad s proměnnou:
	\begin{lstlisting}[language=bash]
  for jmeno in minotaur.dat unicorn.dat
  do
    head -n1 $jmeno
  done
	\end{lstlisting}
			
	\item Příklad - vypsání souborů:
	\begin{lstlisting}[language=bash]
  for soubor in *.dat; do
    echo $soubor
  done
	\end{lstlisting}

	\end{itemize}
\end{frame}

\begin{frame}[fragile]{Cyklus \texttt{for} – vnořené cykly}
	\begin{itemize}
	\item Příklad: Vytvořit složky z názvů molekul a určitých teplot		
	\begin{lstlisting}[language=bash]
for molekula in cubane ethane methane
do
   for teplota in 25 30 37 40
   do
      echo mkdir $molekula-$teplota
   done
done
	\end{lstlisting}
\end{itemize}
\end{frame}

\section{Proměnné prostředí}

\begin{frame}[fragile]{Proměnné prostředí}
	\begin{itemize}
		\item Proměnné bashe - jsou dostupné jen pro samotný Bash
		\item Proměnné prostředí - dostupné i pro běžící programy 
		\vspace{2em}
		\item Příkaz \texttt{env} vypíše všechny proměnné prostředí:
		\begin{lstlisting}[language=bash]
	  $ env | head
	  $ env | grep ^LANG
		\end{lstlisting}
	\end{itemize}
\end{frame}

\begin{frame}[fragile]{Exportování}
	\begin{itemize}
		\item způsob, jak předat proměnnou programům
		\begin{lstlisting}[language=bash]
	$ export jmeno=basilisk.dat
	$ export jmeno
		\end{lstlisting}
		
		\item Příklad
	\end{itemize}
	
	\begin{columns}[T]
		\begin{column}{0.48\textwidth}
			\textbf{Bez exportu:}
			\begin{lstlisting}[language=bash]
jmeno=minotaur.dat
env | grep jmeno   
echo $jmeno
minotaur.dat
			\end{lstlisting}
		\end{column}
		\begin{column}{0.48\textwidth}
			\textbf{S exportem:}
			\begin{lstlisting}[language=bash]
export jmeno
env | grep jmeno   
jmeno=minotaur.dat 
echo $jmeno
minotaur.dat
			\end{lstlisting}
		\end{column}
	\end{columns}
\end{frame}


\begin{frame}[fragile]{Složení názvů proměnných, Výzva}
	\begin{itemize}
		\item Pokud chceš mít text za proměnnou (např. „ovoce“ v „jablko“), použij složené závorky:
		\begin{lstlisting}[language=bash]
	$ majitel=Petr
	$ echo ${majitel}ovo jablko    
		\end{lstlisting}
		\vspace{2em}
		\item Bash má vlastní proměnné používané pro prompt: \texttt{PS1}, \texttt{PS2}.  
		\begin{itemize}
		\item Příklad přenastavení výzvy:
		\begin{lstlisting}[language=bash]
	$ PS1='[0_0]'
		\end{lstlisting}
			\end{itemize}
	\end{itemize}
\end{frame}

\section{Skripty}

\begin{frame}[fragile]{Skripty}
	\begin{itemize}
		\item Skripty jsou jednoduché programy – často spojují existující příkazy.
		\item Ukázka obsahu \texttt{klasifikace}:
		\begin{lstlisting}[language=bash]
	for x in *.dat
	do
	  head -n2 $x | tail -n1
	done
		\end{lstlisting}
	\end{itemize}
\end{frame}

\begin{frame}[fragile]{Spustitelný skript}
	\begin{itemize}
		\item Skript se spouští pomocí Bashe:
		\begin{lstlisting}[language=bash]
    $ bash klasifikace
		\end{lstlisting}
		\vspace{1em}
		\item Aby šel spustit jako příkaz, je třeba:
		\begin{enumerate}
			\item Přidat \textbf{shebang} na první řádek:
			\begin{lstlisting}[language=bash]
	  #! /bin/bash
			\end{lstlisting}
			\item Nastavit spustitelný příznak:
			\begin{lstlisting}[language=bash]
	  $ chmod +x klasifikace
			\end{lstlisting}
			\item Spustit:
			\begin{lstlisting}[language=bash]
	  $ ./klasifikace
			\end{lstlisting}
		\end{enumerate}
	\end{itemize}
\end{frame}


\begin{frame}[fragile]{Argumenty skriptu}
	\begin{itemize}
		\begin{lstlisting}[language=bash]
$ ./klasifikuj minotaur.dat
		\end{lstlisting}
		\item Argumenty z příkazové řádky se předávají pomocí speciálních proměnných
		\begin{itemize}
					\item \texttt{"\$0"} – jméno skriptu
				\item \texttt{"\$1"} – první argument
				\item  \texttt{"\$2"}, \texttt{"\$3"} – další argumenty
		\item \texttt{"\$@"} – všechny argumenty
			\end{itemize}
					\vspace{1em}
			\item Použití ve skriptu:
		\begin{lstlisting}[language=bash]
	#! /bin/bash
	head -n2 "$1" | tail -n1
		\end{lstlisting}

	\end{itemize}
\end{frame}

\begin{frame}[fragile]{Proměnná \texttt{PATH}}
	\begin{itemize}
		\item Systém hledá spustitelné programy v adresářích uvedených v proměnné \texttt{PATH}.
		\begin{lstlisting}
$ echo $PATH
/usr/local/bin:/usr/bin:/bin:/usr/local/sbin
		\end{lstlisting}
		\item Skript přesuneme do složky v \texttt{PATH}, např. \texttt{\$HOME/bin}
		\item Spustíme skript jen názvem:
		\begin{lstlisting}[language=bash]
$ klasifikace *.dat
		\end{lstlisting}
		\item K trvalé změně \texttt{PATH} je potřeba upravit ji v souboru \texttt{.bashrc}
	\end{itemize}
\end{frame}


	
	\section{Kombinace příkazů}
	
\begin{frame}[fragile]{Další kombinace příkazů}
	\begin{itemize}
		\item Příkazy lze řetězit pomocí \texttt{;} – provedou se postupně:
		\begin{lstlisting}
$ echo "Seznam souboru:"; ls
		\end{lstlisting}
		\vspace{1em}
		\item Operátor \texttt{\&\&} – druhý příkaz se provede jen pokud první uspěl:
		\begin{lstlisting}
$ mkdir test && echo "Slozka vytvorena."
		\end{lstlisting}
		\vspace{1em}		
		\item Operátor \texttt{||} – druhý příkaz se provede jen pokud první selhal:
		\begin{lstlisting}
$ cat neexistujici.txt || echo "Soubor nebyl nalezen."
		\end{lstlisting}
	\end{itemize}
\end{frame}

	\section{Programy, aliasy}
	
\section{Programy, funkce a aliasy}

\begin{frame}[fragile]{Jak Bash hledá příkazy}
	\begin{itemize}
		\item Když do Bashi zadáš jméno příkazu (např. \texttt{ls}), Bash hledá v několika druzích příkazů:
		\begin{itemize}
			\item exteníbinární programy v adresářích z \texttt{\$PATH}
			\item zabudované příkazy (built-in) – např. \texttt{cd}, \texttt{export} 
			\item aliasy – zkratky pro jiné příkazy 
			\item funkce shellu – skripty přímo definované v Bashi
		\end{itemize}
		\item Příkaz \texttt{type} ukáže, k jakému typu příkaz patří:
		\begin{lstlisting}[language=bash]
	type cd
	type ls
	type myalias
		\end{lstlisting}
	\end{itemize}
\end{frame}

\begin{frame}[fragile]{Alias – jednoduchá zkratka}
	\begin{itemize}
		\item Alias je zkratka, která nahrazuje jeden příkaz jiným (v rámci shellu).
		\item Definice aliasu:
		\begin{lstlisting}
alias ll='ls -l --color=auto'
alias gs='git status'
		\end{lstlisting}
		\item Použití: když napíšeš \texttt{ll /home}, Bash místo toho spustí \texttt{ls -l --color=auto /home}.
		\item Zrušení aliasu:
		\begin{lstlisting}
unalias gs
		\end{lstlisting}
		\item Poznámka: alias funguje jen na začátku příkazu – nepřepíše části za středníkem či v řetězci příkazů.
	\end{itemize}
\end{frame}

\begin{frame}[fragile]{Funkce v Bashi}
	\begin{itemize}
		\item Funkce je více „robustní“ zkratka – může obsahovat více příkazů, logiku, podmínky.
		\item Definice:
		\begin{lstlisting}[language=bash]
myskupina() {
	echo "Spoustim funkci s argumenty: $@"
	ls -l "$1"
}
		\end{lstlisting}
		\item Volání:  
		\begin{lstlisting}[language=bash]
myskupina /etc
		\end{lstlisting}
		\item Argumenty uvnitř: \texttt{\$1}, \texttt{\$2}, …, počet v \texttt{\$\#}, všechny v \texttt{\$@}. 
		\item Užitečný příklad – kombinace mkdir + cd:
		\begin{lstlisting}[language=bash]
mkcd() {
	mkdir -p "$1" && cd "$1"
}
		\end{lstlisting}
	\end{itemize}
\end{frame}

\begin{frame}[fragile]{Zapamatování aliasů \& funkcí (např. v \texttt{.bashrc})}
	\begin{itemize}
		\item Alias nebo funkci definujeme ručně, ale zmizí po ukončení Bashi.
		\item Řešení: vložit je do souboru \texttt{~/.bashrc}, který se načte při startu shellu:
		\begin{lstlisting}[language=bash]
	alias ll='ls -l --color=auto'
	
	mkcd() {
		mkdir -p "$1" && cd "$1"
	}
		\end{lstlisting}
		\item Po úpravě spustíme:
		\begin{lstlisting}[language=bash]
			source ~/.bashrc
		\end{lstlisting}
		\item Nebo otevřít nový terminál — definice se načtou.
	\end{itemize}
\end{frame}

	
	
	
\end{document}
