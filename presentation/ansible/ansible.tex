\documentclass{beamer}
\usepackage[utf8]{inputenc}
\usepackage[T1]{fontenc}
\usepackage[czech]{babel}

\setbeamerfont{title}{size=\LARGE}

\title{Ansible – automatizace správy serverů}
\author{Eliška Jégrová}
\date{25. 11. 2025}

% pro obrázky a kreslení
\usepackage{tikz}
\usetikzlibrary{positioning}
\usetikzlibrary{shapes.misc}
\usetikzlibrary{shapes.symbols}
\usetikzlibrary{shapes, shapes.geometric, positioning, arrows}
\usepackage{graphicx}
\usepackage{datetime}
\usepackage{svg}
\usepackage{listings}
\usepackage{dirtree}

\usepackage{listings}
\usepackage{xcolor} % pokud chceš barvičky, jinak můžeš vynechat

\lstdefinelanguage{yaml}{
	basicstyle=\ttfamily\small,
	keywords={true,false,null,y,n},
	keywordstyle=\bfseries,
	comment=[l]{\#},
	commentstyle=\itshape,
	stringstyle=\ttfamily,
	morestring=[b]",
	morestring=[b]',
	sensitive=true
}

% vlastní příkaz pro tildy
\newcommand{\ts}{\raisebox{-0.25em}{\textasciitilde}}

\begin{document}
	
\frame{\titlepage}

\begin{frame}{Obsah}
	\tableofcontents
\end{frame}



\section{Úvod}

\begin{frame}{Proč Ansible}
	\begin{itemize}
		\item Dosud: ruční správa přes SSH nebo konzoli
		\item Pro jeden server to stačí, ale co budeme dělat v případě, kdy budeme mít serverů 1000?
			\item Nebo budeme potřebovat zreplikovat specifické nastavení na další server?
			\item Jak rychle dokážet nahradit nefunkční server?
	\end{itemize}
\end{frame}
	
	\begin{frame}{Co je Ansible}
		\begin{itemize}
			\item Ansible je nástroj pro \textbf{automatizaci}:
			\begin{itemize}
				\item \textbf{konfigurační management} – udržování konzistentního nastavení systémů
				(balíčky, služby, konfiguráky, uživatelé)
				\item \textbf{deploy aplikací} – nahrání nové verze, konfigurace,
				restart služeb,
				\item \textbf{orchestrace} více serverů.
			\end{itemize}
			
			\item \textbf{Bezagentový nástroj} (\emph{agentless}):
			\begin{itemize}
				\item na spravovaných serverech není potřeba žádný démon,
				\item používá \textbf{SSH} a Python.		
			\end{itemize}

			\item Typické použití:
			\begin{itemize}
				\item nastavení a správa webových serverů,
				\item vytváření a správa uživatelů,
				\item instalace balíčků,
				\item úprava konfigurací a restart služeb.
			\end{itemize}
		\end{itemize}
	\end{frame}

\begin{frame}{Základní principy Ansible}
	\begin{itemize}
		
		\item V Ansible konfiguraci popisujeme \textbf{deklarativně}:
		\begin{itemize}
			\item \textbf{jaký má být výsledný stav}, ne jednotlivé kroky,
			\item např. „balíček \texttt{httpd} je nainstalovaný“,
			„služba \texttt{httpd} běží a je povolená po startu“.
		\end{itemize}
		
		\item \textbf{Idempotence}:
		\begin{itemize}
			\item opakované spuštění vede ke stejnému stavu:
				\begin{itemize}
					\item při prvním spuštění se něco změní,
					\item při dalším spuštění už není co měnit.
				\end{itemize}
			\item Příklady idempotentního chování:
			\begin{itemize}
				\item \texttt{package: state=present} nainstaluje balíček jen pokud chybí,
			\end{itemize}
		\end{itemize}
		
		\item Konfigurace je psaná v souborech ve formátu \textbf{YAML} \\ (\texttt{.yml, .yaml}).
	\end{itemize}
\end{frame}


	\section{Instalace}
	
	\begin{frame}[fragile]{Instalace Ansible a první playbook}
		\begin{itemize}
			\item Na řídicím serveru (naší virtuálce) nainstalujeme Ansible:
		\end{itemize}
		
		\begin{verbatim}
			# dnf install ansible
		\end{verbatim}
		
		\begin{itemize}
			\item V pracovním adresáři vytvoříme soubor \texttt{setup.yaml}, např.:
		\end{itemize}
		
		\begin{verbatim}
			    - hosts: localhost
			      connection: local
			      become_user: root
			      tasks:
			      - name: Install htop
			        become: yes
			        dnf:
			          state: latest
			          name:
			          - htop
		\end{verbatim}
	\end{frame}
	
	\begin{frame}{Klíčová slova v setup.yaml}
		\begin{itemize}

			\item \texttt{hosts: localhost}
			\begin{itemize}
				\item na které hosty se playbook aplikuje (skupina \texttt{all} = všichni).
			\end{itemize}
			
			\item \texttt{connection: local}
			\begin{itemize}
				\item způsob jak se připojit
			\end{itemize}
			
			\item \texttt{become\_user: root}
			\begin{itemize}
				\item jaký uživatel bude použit
			\end{itemize}

			\item \texttt{tasks:}
			\begin{itemize}
				\item seznam kroků, které se mají provést.
			\end{itemize}
			
			\item \texttt{name: Install htop}
			\begin{itemize}
				\item název konkrétní úlohy (tasku).
			\end{itemize}

			\item \texttt{become: yes}
			\begin{itemize}
				\item úlohy poběží s právy zadaného uživatele.
			\end{itemize}

		\end{itemize}
	\end{frame}
	
	\begin{frame}[fragile]{Spuštění playbooku setup.yaml}
		\begin{itemize}
			\item Playbook spouštíme příkazem \texttt{ansible-playbook}.
		\end{itemize}
		
		\begin{verbatim}
			$ ansible-playbook -K setup.yaml -i hosts.yaml 
		\end{verbatim}
		
		\begin{itemize}
			\item \texttt{-i } – určuje soubor s inventářem (seznam hostů).
			\item \texttt{-K} – Ansible se zeptá na heslo pro \texttt{sudo}
			(become heslo), aby mohl spouštět úlohy s vyššími právy.
			\item Během běhu Ansible ukazuje pro každý task:
			\begin{itemize}
				\item jestli proběhl v pořádku (\texttt{ok}),
				\item jestli něco změnil (\texttt{changed}),
				\item případné chyby (\texttt{failed}).
			\end{itemize}
		\end{itemize}
	\end{frame}

\section{Inventář}
	\begin{frame}[fragile]{Inventář – seznam spravovaných serverů}
		\begin{itemize}
			\item Inventář říká, \textbf{na které stroje} se mají playbooky pouštět.
			\item Každý řádek = jeden server (jméno nebo IP adresa).
		\end{itemize}
		
		\begin{verbatim}
# hosts.yml - jednoduchý inventář v YAML
web:
  hosts:
    192.168.122.101:
    192.168.122.102:
db:
  hosts:
    192.168.122.110:
		\end{verbatim}
		
		\begin{itemize}
			\item Jména \texttt{web} a \texttt{db} jsou \textbf{skupiny} hostů
			\item V \texttt{setup.yaml} pak můžeme použít:\hspace{0.5em} \texttt{hosts: web} \\
			a playbook se spustí na všech web serverech.
		\end{itemize}
	\end{frame}

	\section{Samostatná práce}
	
\begin{frame}{Samostatná práce – první kroky s Ansible}
	\begin{itemize}
		\item Otevři soubor \texttt{setup.yaml},
		\begin{enumerate}

		\item Uprav task tak, aby instaloval víc „užitečných nástrojů“: keepassxc a gnome-tweaks.
		
		\item Spusť playbook:
		\begin{itemize}
			\item \texttt{ansible-playbook -K setup.yaml}
		\end{itemize}
		Sleduj, které balíčky mají stav \texttt{changed} a které jen \texttt{ok}.
		
		\item Do seznamu \texttt{tasks:} přidej úlohu, která:
		\begin{itemize}
			\item v systému vytvoří novou uživatelku \texttt{zuzana}.
			\item použij k tomu modul user
		\end{itemize}

	
		\item Přidej další úlohu, která:
		\begin{itemize}
			\item zajistí, že balíček \texttt{sl} \textbf{není} nainstalovaný
		\end{itemize}
			
		\item Přidej úlohu, která:
		\begin{itemize}
			\item zajistí, že služba \texttt{sshd} běží,
			\item a je \textbf{povolena po startu systému} (enabled).
			\item použij modul systemd
		\end{itemize}
	\end{enumerate}
	\end{itemize}
\end{frame}

\end{document}
