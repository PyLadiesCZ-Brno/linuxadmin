\documentclass{beamer}
\usepackage[utf8]{inputenc}
\usepackage[T1]{fontenc}
\usepackage[czech]{babel}

\setbeamerfont{title}{size=\LARGE}

\title{Kontejnery}
\author{Eliška Jégrová}
\date{28. 11. 2025}

% pro obrázky a kreslení
\usepackage{tikz}
\usetikzlibrary{positioning}
\usepackage{graphicx}
\usepackage{datetime}
\usepackage{svg}
\usepackage{listings}
\usepackage{dirtree}


% vlastní příkaz pro tildy
\newcommand{\ts}{\raisebox{-0.25em}{\textasciitilde}}

\begin{document}
	
	\frame{\titlepage}
	
	\begin{frame}{Obsah}
		\tableofcontents
	\end{frame}

\section{Úvod}

\begin{frame}{It works on my machine!}
	\begin{itemize}
		\item Aplikace často funguje na vývojářově počítači, ale ne na serveru.
		\item Na každém stroji může být jiná verze knihoven, balíčků nebo konfigurace.
		\item Kontejner používá \textbf{vlastní souborový systém} oddělený od zbytku systému:
		\begin{itemize}
			\item obsahuje potřebné programy, knihovny a konfiguraci,
			\item všude vypadá stejně, bez ohledu na konkrétní stroj.
		\end{itemize}
		\item Stejný kontejner tedy můžeme spustit na různých počítačích
		a očekávat \textbf{stejné prostředí i chování aplikace}.
	\end{itemize}
\end{frame}

\begin{frame}{Co je kontejner a jak se liší od virtuálky}
	\begin{itemize}
		\item \textbf{Kontejner}
		\begin{itemize}
			\item běžící proces s vlastním pohledem na souborový systém,
			\item má izolované prostředí (procesy, síť),
			\item sdílí stejný kernel s hostitelským systémem.
		\end{itemize}
		
		\item \textbf{Virtuální stroj}
		\begin{itemize}
			\item běží v něm celý operační systém včetně vlastního kernelu,
			\item emuluje „celý počítač“ (CPU, disk, síťová karta),
			\item je těžší na zdroje a startuje pomaleji.
		\end{itemize}
		
	\end{itemize}
\end{frame}

	\begin{frame}{Sjednocení a izolace}
		\begin{itemize}
			\item \textbf{Sjednocení prostředí}
			\begin{itemize}
				\item stejný obraz kontejneru můžeme použít na více strojích,
				\item ve vývoji, testu i produkci běží aplikace ve stejném prostředí,
				\item odpadá „u mě to funguje, ale na serveru ne“ kvůli rozdílným verzím balíčků.
			\end{itemize}
			
			\item \textbf{Izolace služeb}
			\begin{itemize}
				\item každá služba může běžet ve svém vlastním kontejneru,
				\item chyby nebo špatná konfigurace jedné služby méně ovlivní zbytek systému,
				\item v každém kontejneru může být jiná verze programu nebo knihovny,
				aniž by si navzájem překážely.
			\end{itemize}
		\end{itemize}
	\end{frame}

\begin{frame}{Sdílení vrstev}
	\begin{itemize}
		\item Vrstvy obrazu jsou \textbf{jen pro čtení} a dají se \textbf{sdílet}:
		\begin{itemize}
			\item více obrazů může používat stejné spodní vrstvy,
			\item více kontejnerů z jednoho obrazu sdílí všechny jeho vrstvy.
		\end{itemize}
		
		\item Každý kontejner má navíc jen svou \textbf{zapisovatelnou vrstvu}:
		\begin{itemize}
			\item tam se ukládají změny (nové soubory, logy, dočasná data),
			\item spodní vrstvy zůstávají stejné pro všechny.
		\end{itemize}
		
		\item Výhody sdílení vrstev:
		\begin{itemize}
			\item šetří místo na disku – stejné soubory nejsou uloženy vícekrát,
			\item rychlejší stažení nových obrazů – znovu se stahují jen nové vrstvy,
			\item rychlejší start kontejnerů – většina dat už je lokálně k dispozici.
		\end{itemize}
	\end{itemize}
\end{frame}

\begin{frame}{Vrstvy obrazu a jejich sdílení}
	\begin{itemize}
		\item \textbf{Vrstva} (layer):
		\begin{itemize}
			\item část souborového systému obrazu,
			\item obsahuje změny oproti předchozí vrstvě
			(přidané soubory, úpravy, smazání),
			\item vrstvy se skládají na sebe „jako palačinky“.
		\end{itemize}
		
		\item \textbf{Obraz} kontejneru:
		\begin{itemize}
			\item je složený z několika vrstev,
			\item všechny tyto vrstvy jsou jen pro čtení.
		\end{itemize}
		
		\item \textbf{Kontejner}:
		\begin{itemize}
			\item používá všechny vrstvy obrazu,
			\item navíc má svou \textbf{zapisovatelnou vrstvu} pro změny.
		\end{itemize}
		
		\item \textbf{Sdílení vrstev}:
		\begin{itemize}
			\item více obrazů může sdílet stejné spodní vrstvy,
			\item více kontejnerů ze stejného obrazu sdílí všechny jeho vrstvy,
			\item šetří to místo na disku a zrychluje stahování i start kontejnerů.
		\end{itemize}
	\end{itemize}
\end{frame}


	\begin{frame}{OverlayFS – vrstvy obrazu a kontejneru}
		\centering
		\begin{tikzpicture}[
			layer/.style={draw, thick, minimum width=6cm, minimum height=0.8cm, align=center},
			>=stealth
			]
			
			% image layers (read-only)
			\node[layer, fill=gray!15] (l1) {Vrstva obrazu 1 (read-only)};
			\node[layer, fill=gray!20, above=0.2cm of l1] (l2) {Vrstva obrazu 2 (read-only)};
			\node[layer, fill=gray!25, above=0.2cm of l2] (l3) {Vrstva obrazu 3 (read-only)};
			
			% container writable layer
			\node[layer, fill=blue!15, above=0.4cm of l3] (upper) {Zapisovatelná vrstva kontejneru};
			
			% merged view
			\node[layer, fill=green!15, right=3.5cm of l3] (merged) {Výsledný pohled v kontejneru};
			
			% arrows
			\draw[->, thick] (l1.east) -- (merged.south west);
			\draw[->, thick] (l2.east) -- (merged.west);
			\draw[->, thick] (l3.east) -- (merged.north west);
			\draw[->, thick] (upper.east) -- (merged.north);
			
			% labels
			\node[above left=0.1cm and -0.2cm of upper] {\small overlayfs};
		\end{tikzpicture}
		
		\vspace{1em}
		{\small Spodní vrstvy pocházejí z obrazu, nahoře je zapisovatelná vrstva konkrétního kontejneru.}
	\end{frame}
	
\end{document}