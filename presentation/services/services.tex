\documentclass{beamer}
\usepackage[utf8]{inputenc}
\usepackage[T1]{fontenc}
\usepackage[czech]{babel}

\setbeamerfont{title}{size=\LARGE}

\title{Služby, webový server a ssh}
\author{Eliška Jégrová}
\date{17. 11. 2025}

% pro obrázky a kreslení
\usepackage{tikz}
\usetikzlibrary{positioning}
\usetikzlibrary{shapes.misc}
\usetikzlibrary{shapes.symbols}
\usetikzlibrary{shapes, shapes.geometric, positioning, arrows}
\usepackage{graphicx}
\usepackage{datetime}
\usepackage{svg}
\usepackage{listings}
\usepackage{dirtree}


% vlastní příkaz pro tildy
\newcommand{\ts}{\raisebox{-0.25em}{\textasciitilde}}

\begin{document}
	
	\frame{\titlepage}
	
	\begin{frame}{Obsah}
		\tableofcontents
	\end{frame}
	
	
\section{Úvod do problematiky}

\begin{frame}{Co je počítačová síť}
	\begin{itemize}
		\item Pro komunikaci mezi počítači je potřeba \textbf{síť}.
		\item Umožňuje sdílení dat, provoz síťových služeb.
		\item Základní pojmy:
		\begin{itemize}
			\item \textbf{host} – zařízení v síti,
			\item \textbf{server} – poskytuje službu,
			\item \textbf{klient} – službu využívá.
		\end{itemize}

	\end{itemize}
\end{frame}



\begin{frame}{Síť – host, router, server}
\begin{figure}
	\includegraphics[width=0.8\textwidth]{../../../../../../Downloads/sít_diagram.drawio-1}	
\end{figure}
\end{frame}


\begin{frame}{IP adresa}
	\begin{itemize}
		\item \textbf{IP adresa} = identifikátor zařízení v síti.
		\item Dva hlavní typy:
		\begin{itemize}
			\item \textbf{IPv4} (32 bitů): např. 192.168.1.10
			\item \textbf{IPv6} (128 bitů): např. 2001:db8::1
		\end{itemize}
		\item IP adresa umožní směrovat data na správný počítač.
		\item Často se rozlišuje:
		\begin{itemize}
			\item \textbf{privátní adresa} – v lokální síti,
			\item \textbf{veřejná adresa} – viditelná na internetu.
		\end{itemize}
		\item Příkaz:
		\begin{itemize}
			\item \texttt{\$ ip address}
			\item \texttt{\$ hostname -I} {} – vypíše jen IP adresy hosta
		\end{itemize}
	\end{itemize}
\end{frame}

\begin{frame}{DNS – Domain Name System}
	\begin{itemize}
		\item Lidé si snadněji pamatují názvy než čísla.
		\item \textbf{DNS} převádí názvy (např. \texttt{example.com}) na IP adresy.
		\item Funguje podobně jako telefonní seznam.
		\item Typické příkazy:
		\begin{itemize}
			\item \texttt{\$ ping seznam.cz}
			\item \texttt{\$ dig seznam.cz}
		\end{itemize}
		\item Bez DNS by web nefungoval – museli bychom psát IP adresy ručně.
	\end{itemize}
\end{frame}

\begin{frame}{Porty a služby}
	\begin{itemize}
		\item Jeden server může poskytovat mnoho služeb zároveň.
		\item Aby se služby nepletly, používají \textbf{porty}.
		\item Port = „číslo služby“.
		\item Příklady běžných portů:
		\begin{itemize}
			\item 22 – SSH
			\item 80 – HTTP (web)
			\item 443 – HTTPS (web zabezpečený)
		\end{itemize}
		\item Kombinace: \textbf{IP adresa + port} = konkrétní služba.
		\item Např.: \\[0.3em]
		\texttt{192.168.1.20:80} → webový server  \\
		\texttt{192.168.1.20:22} → SSH \\
		\texttt{192.168.1.20:25} → emailový server
	\end{itemize}
\end{frame}

\begin{frame}{Jak spolu vše souvisí}
	\begin{itemize}
		\item Uživatel zadá do prohlížeče název: \texttt{seznam.cz}.
		\item DNS jej přeloží na IP adresu serveru.
		\item Prohlížeč se spojí na port 80/443.
		\item Server vrátí odpověď – webovou stránku.
		\item SSH funguje podobně, jen na portu 22 a bez webových dat.
	\end{itemize}
	\vspace{3em}
	\begin{tikzpicture}[
	node/.style={
		draw,
		rounded corners,
		thick,
		minimum width=3cm,
		minimum height=1cm,
		align=center
	}
	]
	
	% Nodes
	\node[node] (client) {Klient \\ (např. prohlížeč)};
	\node[node, right=4cm of client] (server) {Server \\ (např. web server)};
	\node[below=1.5cm of client] (note1) {};
	\node[below=1.5cm of server] (note2) {};
	
	% Arrows
	\draw[->, thick] (client) -- node[above]{požadavek} (server);
	\draw[->, thick] (server) -- node[below]{odpověď} (client);		
	\end{tikzpicture}	
\end{frame}

\section{Firewall}

\begin{frame}[fragile]{Firewall}
	\begin{itemize}
		\item Firewall je síťové zařízení nebo software, který \textbf{povoluje nebo blokuje připojení} podle nastavených pravidel.
		\item Slouží k ochraně serveru nebo počítače před nežádoucí komunikací.
		\item V nových instalacích je firewall obvykle \textbf{zapnutý a defaultně blokuje většinu portů}.
		\item Na Fedoře se používá \textbf{firewalld} (dynamický firewall s podporou zón).

		\item Příkaz pro okamžité povolení:
		\begin{verbatim}
		# firewall-cmd --add-service=http
		\end{verbatim}
		\item Příkaz pro trvalé povolení po restartu:
		\begin{verbatim}
		# firewall-cmd --permanent --add-service=http
		\end{verbatim}
	\end{itemize}
\end{frame}

\section{Webový server}

\begin{frame}{Služba a démon}
	\begin{itemize}
		\item \textbf{Služba} (\emph{service}):
		\begin{itemize}
			\item něco, co lze zapnout a vypnout
			\item má stav (běží/neběží), konfiguraci, logy a závislosti,
			\item příklady: webový server, firewall, tiskový server.
		\end{itemize}
		
		\item \textbf{Démon} (\emph{daemon}):
		\begin{itemize}
			\item dlouho běžící proces „na pozadí“, typicky spuštěný službou,
			\item nespouští se z shellu
			\item často má v názvu \texttt{d} na konci: \texttt{httpd}, \texttt{firewalld}, \texttt{sshd}.
		\end{itemize}
		
		\item Na Fedoře služby spravuje \textbf{systemd}  (PID 1); ovládáme je příkazem \texttt{systemctl}.
	\end{itemize}
\end{frame}


\begin{frame}{Webový server httpd}
	\begin{itemize}
		\item Použijeme \textbf{Apache HTTP Server} (balíček \texttt{httpd}).
		\item Běží jako \textbf{démon} – dlouho běžící proces na pozadí.
		\item Naslouchá typicky na portu \textbf{80} (HTTP), případně \textbf{443} (HTTPS).
		\item Webový server běží jako služba \texttt{httpd.service}.
		\item Existují alternativy (např. \texttt{nginx}), ale princip správy služby je podobný.
	\end{itemize}
\end{frame}


\begin{frame}[fragile]{Instalace webového serveru}
	\begin{itemize}
		\item Instalace na Fedoře:
		\begin{verbatim}
			# dnf install httpd
		\end{verbatim}
		\item Balíček obsahuje binárku serveru, konfiguraci i dokumentaci.
		\item Důležité konfigurační soubory:
		\begin{itemize}
			\item \texttt{/etc/httpd/conf/httpd.conf} – hlavní konfigurace,
			\item \texttt{/etc/httpd/conf.d/*.conf} – další části nastavení,
			\item \texttt{/etc/httpd/conf.modules.d/*.conf} – seznam modulů.
		\end{itemize}
	\end{itemize}
\end{frame}

\begin{frame}[fragile]{Start, stop, stav služby}
	\begin{itemize}
		\item Systémové služby se spravují jako \textbf{root} (např. \texttt{\$ sudo -i}).
		\item Základní příkazy:
		\begin{verbatim}
			# systemctl start httpd   - spuštění služby (a démona)
			# systemctl stop httpd    - zastavení služby
			# systemctl status httpd  - informace o běhu
		\end{verbatim}
		\item \texttt{status} ukáže:
		\begin{itemize}
			\item zda služba běží,
			\item PID a počet procesů \texttt{httpd},
			\item poslední hlášky z logu.
		\end{itemize}
	\end{itemize}
\end{frame}

\begin{frame}[fragile]{Test webového serveru}
	\begin{itemize}
		\item Po \texttt{systemctl start httpd} by server měl naslouchat na portu 80.
		\item Test z příkazové řádky:
		\begin{verbatim}
			$ curl http://192.168.1.10
		\end{verbatim}
		\item Textový prohlížeč (ve virtuálce):
		\begin{verbatim}
			# dnf install links
			$ links http://192.168.1.10
		\end{verbatim}
		\item V prohlížeči zadejte do adresního řádku: \texttt{http://192.168.1.10}
		\item Při správně nastavené síti a povolené službě \texttt{http}
		ve firewallu je testovací stránka dostupná i z jiného počítače.
	\end{itemize}
\end{frame}

\begin{frame}[fragile]{Automatické spuštění po startu}
	\begin{itemize}
		\item Na Fedoře se nově nainstalované služby \textbf{nespouští automaticky}.
		\item Nastavení spuštění při startu systému:
		\begin{verbatim}
			# systemctl enable httpd   # zapnout při startu
			# systemctl disable httpd  # nezapínat při startu
		\end{verbatim}
		\item Ověření:
		\begin{itemize}
			\item restart systému,
			\item znovu \texttt{systemctl status httpd}.
		\end{itemize}
	\end{itemize}
\end{frame}

\begin{frame}[fragile]{Změna konfigurace a reload}
	\begin{itemize}
		\item Konfigurační soubory (např. \texttt{/etc/httpd/conf.d/welcome.conf})
		se načítají při startu služby.
		\item Po změně konfigurace je potřeba server informovat:
		\begin{verbatim}
			# systemctl reload httpd   # načtení nové konfigurace
			# systemctl restart httpd  # úplný restart služby
		\end{verbatim}
		\item \texttt{reload}:
		\begin{itemize}
			\item běžící spojení zůstanou zachována,
			\item nová spojení už používají nové nastavení.
		\end{itemize}
		\item \texttt{restart}:
		\begin{itemize}
			\item služba se ukončí a znovu spustí,
			\item aktuální spojení se zavřou.
		\end{itemize}
	\end{itemize}
\end{frame}

\begin{frame}[fragile]{DocumentRoot a obsah webu}
	\begin{itemize}
		\item \textbf{DocumentRoot} = adresář, odkud server servíruje soubory.
		\item Nastavení je v \texttt{/etc/httpd/conf/httpd.conf}
		\item Výchozí DocumentRoot na Fedoře (Apache): \texttt{/var/www/html/}.
		\item Vytvoř testovací soubor:
		\begin{verbatim}
			# echo "Hello, World" > /var/www/html/hello.txt
		\end{verbatim}
		\item Otevři v prohlížeči:
		\begin{verbatim}
			http://192.168.1.10/hello.txt
		\end{verbatim}
		\item Změny v \textbf{obsahu} DocumentRootu se projeví hned
		– není potřeba \texttt{reload}.
	\end{itemize}
\end{frame}



\begin{frame}[fragile]{Logy webového serveru}
	\begin{itemize}
		\item Pro hledání chyb a kontrolu běhu slouží logy.
		\item Základní přehled: \texttt{systemctl status httpd}.
		\item Detailnější log přes \texttt{journalctl}:
		\begin{verbatim}
			$ journalctl -u httpd
		\end{verbatim}
		\item V logu najdeš např.:
		\begin{itemize}
			\item start/stop služby,
			\item chybová hlášení,
			\item informaci, na jakém portu server naslouchá.
		\end{itemize}
	\end{itemize}
\end{frame}

\begin{frame}[fragile]{Signály a démoni}
	\begin{itemize}
		\item Každá služba může reagovat na signály trochu jinak.
		\item \textbf{systemd unit} popisuje, jak se služba spouští, zastavuje a reloaduje.
		\item Jak to zjistit:
		\begin{itemize}
			\item zobrazit definici služby:
			\begin{verbatim}
				$ systemctl cat httpd
				$ systemctl cat sshd
			\end{verbatim}
		\end{itemize}
		\item Z toho pak vyčteš:
		\begin{itemize}
			\item jestli \texttt{reload} jen načte konfiguraci,
			\item jak „elegantně“ se služba ukončuje,
			\item jaké signály daný démon pro tyto akce používá.
		\end{itemize}
	\end{itemize}
\end{frame}

\begin{frame}{Samostatná práce 1 – služby a webový server}
	\begin{enumerate}
		\item Ověř, zda je na virtuálce nainstalovaný balíček \texttt{httpd}.
		Pokud ne, nainstaluj ho.
		
		\item Zjisti, jestli se služba \texttt{httpd} spouští automaticky po startu
		systému. Pokud ne, nastav, aby se spouštěla.
		
		\item Vytvoř jednoduchou stránku \texttt{status.html} v DocumentRootu
		(\texttt{/var/www/html/}), která bude obsahovat:
		\begin{itemize}
			\item jméno serveru (např. Virtuálka),
			\item tvoje jméno,
			\item k čemu tento server slouží.
		\end{itemize}
		
		\item V prohlížeči na hostitelském systému otevři:
		\texttt{http://IP\_TVE\_VIRTUALKY/status.html}
		a ověř, že se stránka načte.    
	\end{enumerate}
\end{frame}

\begin{frame}{Samostatná práce 2 – úpravy httpd.conf}
	\begin{enumerate}
		\item V \texttt{/var/www/html} vytvoř soubor:
		\begin{itemize}
			\item \texttt{uvod.html} – krátký text o serveru,
		\end{itemize}
		
		\item V \texttt{httpd.conf} nastav direktivu: \\
		\texttt{DirectoryIndex uvod.html index.html}.\\[0.3em]
		Ověř v prohlížeči, že se při otevření
		\texttt{http://IP\_SERVERU/} načte právě \texttt{uvod.html}.
		
		\item Bonus: Přidej vlastní chybovou stránku v \texttt{httpd.conf}: \\
		\texttt{ErrorDocument 404 /chyba-404.html}.\\[0.3em]
		Vytvoř \texttt{/var/www/html/chyba-404.html} a ověř,
		že se zobrazí při zadání neexistující URL (např. testuji.html).
		
		\item Bonus: Zkontroluj, že v logu není chyba konfigurace.
	\end{enumerate}
\end{frame}

\section{SSH – vzdálená administrace}

\begin{frame}{SSH – vzdálená administrace}
	\begin{itemize}
		\item \textbf{SSH} = \emph{Secure Shell} – bezpečné vzdálené přihlášení na server.
		\item Správa serveru „jako přes terminál“, ale přes síť / internet.
		\item Veškerá komunikace (včetně hesla) je \textbf{šifrovaná}.
		\item Strany spojení:
		\begin{itemize}
			\item \textbf{server} – služba \texttt{sshd} běžící na vzdáleném stroji,
			\item \textbf{klient} – program \texttt{ssh} nebo grafický klient (např. PuTTY).
		\end{itemize}
		\item Typické použití:
		\begin{itemize}
			\item přihlášení na server,
			\item spouštění příkazů, správa služeb, editace konfigurace.
		\end{itemize}
	\end{itemize}
\end{frame}

\begin{frame}[fragile]{Instalace SSH serveru (OpenSSH)}
	\begin{itemize}
		\item Na Fedoře se používá implementace \textbf{OpenSSH}.
		\item Instalace serveru:
		\begin{verbatim}
			$ sudo dnf install openssh-server
		\end{verbatim}
		\item Spuštění služby \texttt{sshd}:
		\begin{verbatim}
			$ sudo systemctl start sshd
		\end{verbatim}
		\item Otevření ve firewallu (port 22):
		\begin{verbatim}
			$ sudo firewall-cmd --add-service=ssh
		\end{verbatim}
		\item Po nastavení můžeš z jiného stroje používat SSH přihlášení.
	\end{itemize}
\end{frame}

\begin{frame}[fragile]{První připojení – localhost}
	\begin{itemize}
		\item Nejdřív si SSH vyzkoušíme „sama na sebe“:
		\begin{verbatim}
			$ ssh localhost
		\end{verbatim}
		\item \texttt{localhost} = „tento počítač“ (bez ohledu na IP).
		\item Při prvním připojení:
		\begin{itemize}
			\item SSH vypíše \textbf{otisk (fingerprint)} klíče serveru,
			\item zeptá se, zda důvěřuješ tomuto serveru (\texttt{yes/no}),
			\item po potvrzení a zadání hesla jsi přihlášená.
		\end{itemize}
		\item Otisk se dá na serveru zjistit (správce posílá přes jiný kanál):
		\begin{verbatim}
			$ sudo ssh-keygen -l -f /etc/ssh/ssh_host_ecdsa_key.pub
		\end{verbatim}
		\item Po přihlášení vidíš shell na vzdáleném stroji,
		prompt bude např. \texttt{petr@localhost:\ts\$}.
	\end{itemize}
\end{frame}

\begin{frame}[fragile]{Soubor \texttt{known\_hosts} a změna klíče}
	\begin{itemize}
		\item Po prvním připojení si klient SSH uloží klíč serveru do:

			\texttt{\ts/.ssh/known\_hosts}

		\item Při dalším připojení kontroluje, zda se klíč nezměnil.
		\item Když se klíč změní, SSH varuje
		(může to znamenat útok typu „man-in-the-middle“).
		\item Postup:
		\begin{itemize}
			\item ověř u administrátora, že klíč byl skutečně změněn,
			\item pokud je vše v pořádku, smaž starý řádek z \texttt{known\_hosts},
			\item připoj se znovu a nový otisk znovu potvrď.
		\end{itemize}
	\end{itemize}
\end{frame}

\begin{frame}[fragile]{Připojení na vzdálený server}
	\begin{itemize}
		\item Typický tvar příkazu:
		\begin{verbatim}
			$ ssh uzivatel@192.168.122.133
		\end{verbatim}
		\item Uživatelské jméno bude jiné než na „tvém“ počítači.
	\end{itemize}
\end{frame}


\begin{frame}[fragile]{Jméno serveru}
	\begin{itemize}
		\item Pro orientaci je dobré mít na serveru \textbf{smysluplné jméno}.
		\item Zobrazení jména:
		\begin{verbatim}
			$ hostname
			$ hostnamectl
		\end{verbatim}
		\item Dočasná změna (do restartu):
		\begin{verbatim}
			$ sudo hostname virtualka
		\end{verbatim}
		\item Trvalé nastavení je v souboru \texttt{/etc/hostname}
		\item Permanentní změna:
		\begin{verbatim}
			$ sudo hostnamectl set-hostname virtualka
		\end{verbatim}
	\end{itemize}
\end{frame}

\begin{frame}{SSH klíče – motivace}
	\begin{itemize}
		\item Přihlašování \textbf{heslem}:
		\begin{itemize}
			\item jednoduché na začátek,
			\item méně pohodlné a méně bezpečné (hesla se dají hádat).
		\end{itemize}
		\item Přihlašování pomocí \textbf{SSH klíčů}:
		\begin{itemize}
			\item používá se dvojice \textbf{veřejný} + \textbf{soukromý} klíč,
			\item veřejný klíč je na serveru, soukromý máš jen ty,
			\item po nastavení se už k serveru přihlašuješ bez zadávání hesla
			k účtu (jen případné heslo ke klíči).
		\end{itemize}
		\item Výhoda:
		\begin{itemize}
			\item bezpečnější než hesla,
			\item pohodlnější při častém připojování (skripty, Git, automatizace).
		\end{itemize}
	\end{itemize}
\end{frame}

\begin{frame}[fragile]{SSH klíče – vytvoření a použití}
	\begin{itemize}
		\item Vytvoření klíčů (na svém počítači):
		\begin{verbatim}
			$ ssh-keygen
		\end{verbatim}
		\item Výchozí umístění:
		\begin{itemize}
			\item soukromý klíč: \texttt{\ts{}/.ssh/id\_rsa}
			\item veřejný klíč: \texttt{\ts{}/.ssh/id\_rsa.pub}
		\end{itemize}
		\item Veřejný klíč se kopíruje na server do
		\texttt{\ts/.ssh/authorized\_keys}.
		\item Pohodlný způsob: \texttt{ssh-copy-id}:
		\begin{verbatim}
			$ ssh-copy-id uzivatel@server
		\end{verbatim}
		\item Po úspěchu se můžeš přihlásit:
		\begin{verbatim}
			$ ssh uzivatel@server
		\end{verbatim}
		\item Pokud má klíč nastavené heslo, zadáš jen to.
	\end{itemize}
\end{frame}

\begin{frame}[fragile]{Nastavení SSHD}
	\begin{itemize}
		\item Konfigurace serveru \texttt{sshd}:
		\begin{itemize}
			\item hlavní soubor: \texttt{/etc/ssh/sshd\_config},
			\item další části: \texttt{/etc/ssh/sshd\_config.d/}.
		\end{itemize}
		\item Lze měnit např.:
		\begin{itemize}
			\item port (např. z 22 na 2222),
			\item povolené metody přihlášení (hesla/klíče),
			\item verze protokolu, povolené šifry atd.
		\end{itemize}
		\item Při změně portu nezapomeň na:
		\begin{itemize}
			\item firewall,
			\item parametr \texttt{-p} na straně klienta.
		\end{itemize}
	\end{itemize}
\end{frame}

\begin{frame}{Samostatná práce 3 – SSH a hostname}
	\begin{enumerate}
		\item Zkontroluj, zda je nainstalovaný a spuštěný \textbf{SSH server}:
		\begin{itemize}
			\item pokud neběží, spusť ho a nastav automatické spuštění po startu.
		\end{itemize}
		\item Nastav na virtuálce hostname, např. \texttt{web-vm1} (hostnamectl)
		Pomocí odhlášení/přihlášení ověř, že se jméno objeví v promptu.
		

		
		\item Z hostitelského systému se přihlas na server
		a spusť tam příkaz:
		\begin{itemize}
			\item \texttt{hostname; whoami; uptime}
		\end{itemize}
		
		\item Najdi ve svém domovském adresáři soubor \texttt{.ssh/known\_hosts}
		a podívej se, jak je v něm tvůj server uložený (IP, jméno, typ klíče).
	\end{enumerate}
\end{frame}

\begin{frame}{Samostatná práce 4 (BONUS) – kopírování přes SSH (scp)}
	\begin{enumerate}
		
		\item Na virtuálce v domovském adresáři vytvoř soubor
		\texttt{poznamky.txt} s libovolným obsahem.
		
		\item Z hostitelského systému zkopíruj soubor z virtuálky k sobě:
		\begin{itemize}
			\item \texttt{scp uzivatel@IP\_VIRTUALKY:\ts/poznamky.txt .}
		\end{itemize}
		Ověř, že se soubor objevil v aktuálním adresáři na hostiteli.
		
		\item Vytvoř na hostitelském systému soubor
		\texttt{readme.txt} a zkopíruj ho na virtuálku
		do domovského adresáře:
		\begin{itemize}
			\item \texttt{scp readme.txt uzivatel@IP\_VIRTUALKY:\ts/}
		\end{itemize}
		Na závěr se na virtuálku přihlas přes SSH a ověř, že tam
		oba soubory opravdu jsou.
	\end{enumerate}
\end{frame}

\section{Cron}

\begin{frame}[fragile]{Co je cron}
	\begin{itemize}
		\item \textbf{cron} je služba, která spouští úlohy v definovaných časech.  
		\item Úlohy se zapisují do souborů zvaných \emph{crontab}.  
		\item Umožňuje plánovat opakující se příkazy:

		\begin{itemize}
		\item Zálohování databáze  
		\item Rotace logů  
		\item Kontrola stavu služby  
		\item Čištění dočasných souborů  
		\item Spouštění skriptů pro sběr dat v určitých intervalech  
		\end{itemize}
	\end{itemize}
\end{frame}

\begin{frame}[fragile]{Syntaxe crontab}
	\begin{itemize}
		\item Každý řádek má 5 polí pro čas + příkaz:
	\end{itemize}
	\begin{verbatim}
		* * * * *  příkaz
		| | | | |
		| | | | |-- den v týdnu (0–6)
		| | | |-- měsíc (1–12)
		| | |-- den v měsíci (1–31)
		| |-- hodina (0–23)
		|-- minuta (0–59)
	\end{verbatim}
	\vspace{0.5em}
	\hspace{2em}\texttt{\$ crontab -e} \\
	\hspace{2em}\texttt{\$ crontab -l}  
\end{frame}



\begin{frame}[fragile]{Příklad cron úlohy}
	\begin{itemize}
		\item Spustit skript každý den ve 2:30 ráno  
	\end{itemize}
	\vspace{0.3em}
	\hspace{2em}\texttt{30 2 * * * /home/uzivatel/backup.sh} \\
	\hspace{2em}– spustí se `backup.sh` každý den v 2:30  
	\vspace{0.5em}
	\begin{itemize}
		\item Spustit skript každou minutu 
	\end{itemize}
	\vspace{0.3em}
	\hspace{2em}\texttt{* * * * * mplayer} \\ \hspace{2.5em}\texttt{/usr/share/sounds/gnome/default/alerts/hum.ogg} \\

	\vspace{0.5em}
	\begin{itemize}
		\item Pro přesměrování výstupu a chyb:  
	\end{itemize}
	\hspace{2em}\texttt{30 2 * * * /path/to/script.sh >} \\ \hspace{2.5em}\texttt{/var/log/muj.log 2>\&1}
\end{frame}

\begin{frame}[fragile]{Samostatná práce 5 – cron}
	\begin{itemize}
		\item Vytvoř skript \texttt{\ts{}/check\_mem.sh}, který vypíše
		aktuální datum a obsazení paměti:
		\begin{verbatim}
			#!/bin/bash
			date
			free -m
		\end{verbatim}
		
		\item Nastav mu spustitelný příznak:

			\texttt{\$ chmod +x \ts/check\_disk.sh}

		
		\item Pomocí \texttt{crontab -e} naplánuj, aby se skript spouštěl
		každých 5 minut a výstup zapisoval do souboru
		\texttt{\ts{}/cron\_mem.log}. Můžeš využít: https://crontab.guru/
		
		\item Počkej, až cron úlohu několikrát spustí, a zkontroluj obsah
		\texttt{cron\_mem.log}. Ověř, že jsou v něm různé časy a aktuální
		využití disku.
	\end{itemize}
\end{frame}




\end{document}