\documentclass{beamer}
\usepackage[utf8]{inputenc}
\usepackage[T1]{fontenc}

\setbeamerfont{title}{size=\LARGE}

\title{Procesy}
\author{Eliška Jégrová}
\date{20. 10. 2025}

% pro obrázky a kreslení
\usepackage{tikz}
\usetikzlibrary{positioning}
\usepackage{graphicx}
\usepackage{datetime}
\usepackage{svg}
\usepackage{listings}


% vlastní příkaz pro tildy
\newcommand{\ts}{\raisebox{-0.25em}{\textasciitilde}}


\begin{document}
	
	\frame{\titlepage}
	
	\begin{frame}{Obsah}
		\tableofcontents
	\end{frame}
	
	
	\section{Návratová hodnota příkazů}

\begin{frame}[fragile]{Návratová hodnota příkazů}
	\begin{itemize}
		\item Každý příkaz po dokončení vrátí číslo — tzv. \texttt{exit code}.
		\item \texttt{0} = úspěch, jiná hodnota = chyba nebo jiný stav.
		\item Poslední návratovou hodnotu zobrazí \texttt{\$?}
	\end{itemize}
	
	\vspace{0.5em}
	\hspace{2em}\texttt{\$ echo Ahoj} \\
	\hspace{2em}Ahoj \\
	\hspace{2em}\texttt{\$ echo \$?} \\
	\hspace{2em}0 \\[0.7em]
	
	\hspace{2em}\texttt{\$ ls neexistuje.txt} \\
	\hspace{2em}ls: cannot access \textquotesingle neexistuje.txt\textquotesingle: No such file or directory \\
	\hspace{2em}\texttt{\$ echo \$?} \\
	\hspace{2em}2
\end{frame}

\section{Procesy}

\begin{frame}[fragile]{Co je proces}
	\begin{itemize}
		\item Proces je termín pro běžící program
		\item Každý proces má své PID (Process ID).
		\item Procesy lze zobrazit příkazem \texttt{ps}.
	\end{itemize}
	\vspace{0.5em}
	\hspace{2em}\texttt{\$ ps} \\
	\hspace{2em}PID\hspace{3em}TTY\hspace{2em}TIME\hspace{2em}CMD \\
	\hspace{2em}2180\hspace{2em}pts/0\hspace{1em}00:00:00 bash \\
	\hspace{2em}2215\hspace{2em}pts/0\hspace{1em}00:00:00 ps
	
	\vspace{2.5em}	
	\hspace{2em}Všechny běžící procesy zobrazí: \texttt{ps -A}
\end{frame}

\begin{frame}[fragile]{Sledování procesů pomocí \texttt{top}}
	\begin{itemize}
		\item Příkaz \texttt{top} zobrazuje procesy v reálném čase.
		\item Zobrazuje vytížení CPU, paměti a běžící procesy.
		\item Ukončení programu: klávesa \texttt{q}.
	\end{itemize}
	\vspace{0.5em}
	\hspace{2em}\texttt{\$ top} \\[0.3em]
	\texttt{top - 09:42:01 up 1:03, 1 user, load average: 0.10, 0.08, 0.02}\\
	\texttt{PID USER \hspace{0.5em}PR \hspace{0.5em}NI \hspace{0.5em}VIRT \hspace{0.5em}RES \hspace{0.5em}S \hspace{0.5em}\%CPU \hspace{0.5em}\%MEM \hspace{0.5em}COMMAND}\\
	\texttt{2180 pepa \hspace{0.6em}20 \hspace{0.5em}0 \hspace{0.1em} 1234 \hspace{0.5em}532 \hspace{0.5em}S \hspace{0.5em}0.0 \hspace{0.5em}0.1 \hspace{0.5em}bash}\\
	\texttt{2250 pepa \hspace{0.6em}20 \hspace{0.5em}0 \hspace{0.5em} 980 \hspace{0.5em}400 \hspace{0.5em}S \hspace{0.5em}0.0 \hspace{0.5em}0.0 \hspace{0.5em}sleep}
\end{frame}

\begin{frame}[fragile]{Ukončení procesu}
	\begin{itemize}
		\item Proces lze ukončit příkazem \texttt{kill}.
		\item Ukončení podle PID: \texttt{kill <pid>}
		\item Násilné ukončení podle PID: \texttt{kill -9 <pid>}
	\end{itemize}
	\vspace{0.5em}
	\hspace{2em}\texttt{\$ ps} \\
	\hspace{2em}2250 pts/0 00:00:00 top \\
	\hspace{2em}\texttt{\$ kill 2250} \\
\end{frame}

\begin{frame}{Samostatná práce – procesy}
	\small
	\begin{enumerate}
		\item Zapni si dva terminály, v prvním spusť \texttt{top}.
		\item Ve druhém terminálu zjisti PID procesu \texttt{top}. (ps)
		\item Z druhého terminálu ukonči proces \texttt{top}. (kill)
	\end{enumerate}
\end{frame}
	
\end{document}