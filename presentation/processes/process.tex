\documentclass{beamer}
\usepackage[utf8]{inputenc}
\usepackage[T1]{fontenc}

\setbeamerfont{title}{size=\LARGE}

\title{Procesy, Soubory a Signály}
\author{Eliška Jégrová}
\date{20. 10. 2025}

% pro obrázky a kreslení
\usepackage{tikz}
\usetikzlibrary{positioning}
\usepackage{graphicx}
\usepackage{datetime}
\usepackage{svg}
\usepackage{listings}


% vlastní příkaz pro tildy
\newcommand{\ts}{\raisebox{-0.25em}{\textasciitilde}}


\begin{document}
	
	\frame{\titlepage}
	
	\begin{frame}{Obsah}
		\tableofcontents
	\end{frame}
	
	
	\section{Návratová hodnota příkazů}

\begin{frame}[fragile]{Návratová hodnota příkazů}
	\begin{itemize}
		\item Každý příkaz po dokončení vrátí číslo — tzv. \texttt{exit code}.
		\item \texttt{0} = úspěch, jiná hodnota = chyba nebo jiný stav.
		\item Poslední návratovou hodnotu zobrazí \texttt{\$?}
	\end{itemize}
	
	\vspace{0.5em}
	\hspace{2em}\texttt{\$ echo Ahoj} \\
	\hspace{2em}Ahoj \\
	\hspace{2em}\texttt{\$ echo \$?} \\
	\hspace{2em}0 \\[0.7em]
	
	\hspace{2em}\texttt{\$ ls neexistuje.txt} \\
	\hspace{2em}ls: cannot access \textquotesingle neexistuje.txt\textquotesingle: No such file or directory \\
	\hspace{2em}\texttt{\$ echo \$?} \\
	\hspace{2em}2
\end{frame}
\begin{frame}{Samostatná práce – návratové hodnoty příkazů}
	\begin{enumerate}
		\item Vytvoř soubor s názvem \texttt{prvni.txt}.
		\item Zkontroluj, zda existuje soubor \texttt{prvni.txt} (ls) 
		\item Vypiš jeho návratovou hodnotu (\texttt{\$?}).
		\item Zkus spustit příkaz \texttt{grep "něco" soubor.txt}
		\item Vypiš návratovou hodnotu.
	\end{enumerate}
\end{frame}

\section{Procesy}

\begin{frame}[fragile]{Co je proces}
	\begin{itemize}
		\item Proces je termín pro běžící program
		\item Každý proces má své PID (Process ID).
		\item Procesy lze zobrazit příkazem \texttt{ps}.
	\end{itemize}
	\vspace{0.5em}
	\hspace{2em}\texttt{\$ ps} \\
	\hspace{2em}PID\hspace{3em}TTY\hspace{2em}TIME\hspace{2em}CMD \\
	\hspace{2em}2180\hspace{2em}pts/0\hspace{1em}00:00:00 bash \\
	\hspace{2em}2215\hspace{2em}pts/0\hspace{1em}00:00:00 ps
	
	\vspace{2.5em}	
	\hspace{2em}Všechny běžící procesy zobrazí: \texttt{ps -A}
\end{frame}

\begin{frame}[fragile]{Sledování procesů pomocí \texttt{top}}
	\begin{itemize}
		\item Příkaz \texttt{top} zobrazuje procesy v reálném čase.
		\item Zobrazuje vytížení CPU, paměti a běžící procesy.
		\item Ukončení programu: klávesa \texttt{q}.
	\end{itemize}
	\vspace{0.5em}
	\hspace{2em}\texttt{\$ top} \\[0.3em]
	\texttt{top - 09:42:01 up 1:03, 1 user, load average: 0.10, 0.08, 0.02}\\
	\texttt{PID USER \hspace{0.5em}PR \hspace{0.5em}NI \hspace{0.5em}VIRT \hspace{0.5em}RES \hspace{0.5em}S \hspace{0.5em}\%CPU \hspace{0.5em}\%MEM \hspace{0.5em}COMMAND}\\
	\texttt{2180 pepa \hspace{0.6em}20 \hspace{0.5em}0 \hspace{0.1em} 1234 \hspace{0.5em}532 \hspace{0.5em}S \hspace{0.5em}0.0 \hspace{0.5em}0.1 \hspace{0.5em}bash}\\
	\texttt{2250 pepa \hspace{0.6em}20 \hspace{0.5em}0 \hspace{0.5em} 980 \hspace{0.5em}400 \hspace{0.5em}S \hspace{0.5em}0.0 \hspace{0.5em}0.0 \hspace{0.5em}sleep}
\end{frame}

\begin{frame}[fragile]{Ukončení procesu}
	\begin{itemize}
		\item Proces lze ukončit příkazem \texttt{kill}.
		\item Ukončení podle PID: \texttt{kill <pid>}
		\item Násilné ukončení podle PID: \texttt{kill -9 <pid>}
	\end{itemize}
	\vspace{0.5em}
	\hspace{2em}\texttt{\$ ps} \\
	\hspace{2em}2250 pts/0 00:00:00 top \\
	\hspace{2em}\texttt{\$ kill 2250} \\
\end{frame}

\begin{frame}{Samostatná práce – procesy}
	\begin{enumerate}
		\item Zapni si dva terminály, v prvním spusť \texttt{top}.
		\item Ve druhém terminálu zjisti PID procesu \texttt{top}. (ps)
		\item Z druhého terminálu ukonči proces \texttt{top}. (kill)
	\end{enumerate}
\end{frame}
	
\begin{frame}[fragile]{Práce s procesy v popředí a na pozadí}
	\begin{itemize}
		\item Proces může běžet:
		\begin{itemize}
			\item \textbf{v popředí} (foreground) – blokuje terminál, dokud neskončí,
			\hspace{2em}\texttt{\$ sleep 100} \\
			\item \textbf{na pozadí} (background) – běží zároveň s ostatními příkazy.
			\hspace{2em}\texttt{\$ sleep 100 \&} \\
			\hspace{1em}[1] 1234  \textit{← číslo úlohy a PID procesu} \\[0.4em]
		\end{itemize}
		\vspace{1em}
		\item Zobrazit běžící a pozastavené úlohy \\
		\hspace{2em}\texttt{\$ jobs} \\
		\item Spustit proces v popředí \\
		\hspace{2em}\texttt{\$ fg} \\
		\item Spustit proces v pozadí \\
		\hspace{2em}\texttt{\$ bg} \\
	\end{itemize}
	
\end{frame}

\begin{frame}{Samostatná práce – práce s procesy}
	\begin{enumerate}
		\item Spusť: \texttt{\$ sleep 1000 \&} 
		\item Podívej se, že proces běží na pozadí. (jobs)
		\item Přesuň úlohu do popředí. (fg)
		\item Zastav běžící proces s \texttt{Ctrl+Z}
		\item Zkontroluj, že je proces zastavený. (jobs)
		\item Spusť proces na pozadí. (bg)
		\item Zkontroluj, že proces běží na pozadí. (jobs, ps)
	\end{enumerate}
\end{frame}

\section{Soubory}

\begin{frame}[fragile]{Co je soubor}
	\begin{itemize}
		\item Soubor je něco, z čeho můžeme číst a/nebo do čeho můžeme zapisovat.
	\end{itemize}

	\vspace{1.8em}
	\textbf{Druhy souborů co známe:}
	\begin{itemize}
		\item Normální soubor uložený na disku.
	\end{itemize}
	\hspace{2em}\texttt{\$ ps -Af > vystup.txt} \\

	\begin{itemize}			
		\item Terminál – soubor reprezentující klávesnici a obrazovku.
		\item Roura – přenos dat mezi procesy.
	\end{itemize}
	
	\vspace{0.5em}
	\hspace{2em}\texttt{\$ ps -Af | grep -w ps} \\
	\hspace{2em}– výstup jednoho procesu proudí do druhého
\end{frame}

\begin{frame}[fragile]{Otevřené soubory procesu}
	\begin{itemize}
		\item Každý proces si může otevírat další soubory – např. pro čtení či zápis.
		\item příkaz \texttt{lsof} - zjistíme, jaké soubory proces používá.
		\item Proměnná \texttt{\$\$} obsahuje PID (číslo aktuálního procesu Bashu).
	\end{itemize}
	
	\vspace{0.2em}
	\hspace{2em}\texttt{\$ echo \$\$} \\
	\hspace{2em}5236 \\[0.6em]
	
	\vspace{0.8em}
	\textbf{Příklad zkráceného výstupu:} \hspace{1em}\texttt{\$ lsof -p \$\$} \\
	\begin{verbatim}
		COMMAND  PID USER  FD   TYPE DEVICE  NAME
		bash    5236 user  cwd  DIR   8,1    /home/user
		bash    5236 user  txt  REG   8,1    /bin/bash
		bash    5236 user    0u  CHR  136,0  /dev/pts/0
		bash    5236 user    1u  CHR  136,0  /dev/pts/0
		bash    5236 user    2u  CHR  136,0  /dev/pts/0
	\end{verbatim}
\end{frame}


\begin{frame}[fragile]{Terminál jako soubor}
	\begin{itemize}
		\item Terminál má svůj název, např. \texttt{/dev/pts/0}.
			\end{itemize}
		\hspace{2em}V předchozím výpisu se jedná o otevřené soubory \texttt{0u, 1u, 2u}.
	\begin{itemize}		
		\item Je to běžný soubor, do kterého lze zapisovat.
	\end{itemize}
	\vspace{0.5em}
	\hspace{2em}\texttt{\$ echo abc > /dev/pts/0}
\end{frame}

\begin{frame}[fragile]{Chybový výstup}
	\begin{itemize}
		\item Každý proces má:
		\begin{itemize}
			\item \texttt{0} – standardní vstup,
			\item \texttt{1} – standardní výstup,
			\item \texttt{2} – chybový výstup.
		\end{itemize}
	\end{itemize}
	\vspace{0.4em}
	\hspace{2em}\texttt{\$ cp a b} \\
	\hspace{2em}\texttt{cp: cannot stat 'a': No such file or directory} \\[0.4em]
	\hspace{2em}\texttt{\$ cp a b > jiny.txt} \\
	\hspace{2em}– chyba se vypíše stále do terminálu
\end{frame}

\begin{frame}[fragile]{Přesměrování chybového výstupu}
	\begin{itemize}
		\item Přesměrování chybového výstupu: \texttt{2>}.
	\end{itemize}
	\vspace{0.4em}
	\hspace{2em}\texttt{\$ cp a b 2> jiny.txt} \\[0.4em]
	\begin{itemize}
		\item Přesměrování obojího:
	\end{itemize}
	\hspace{2em}\texttt{\$ ls existuje.txt neexistuje.txt > vystup.txt 2> chyby.txt} \\[0.4em]
	\begin{itemize}
		\item Nebo obojí do jednoho souboru:
	\end{itemize}
	\hspace{2em}\texttt{\$ ls existuje.txt neexistuje.txt > jiny.txt 2>\&1}
\end{frame}

\begin{frame}[fragile]{Zahození výstupu}
	\begin{itemize}
		\item Soubor \texttt{/dev/null} zahazuje vše, co se do něj zapíše.
	\end{itemize}
	\vspace{0.5em}
	\hspace{2em}\texttt{\$ ls existuje.txt neexistuje.txt 2> /dev/null} \\[0.3em]
	\hspace{2em}– chybové hlášky se „vyhodí“ \\
	\vspace{0.5em}
	\hspace{2em}\texttt{\$ find /var/cache > /dev/null} \\
	\hspace{2em}– zobrazí jen chyby (Permission denied)
\end{frame}
	
\begin{frame}{Samostatná práce – práce se soubory}
	\begin{enumerate}
		\item Zobraz si otevřené soubory procesu bash (lsof, proměnná \$\$).
		\item Zjisti si z výpisu, jak se jmenuje soubor pro terminál.
		\item Zapiš nějaký řetězec do souboru terminálu (echo).
	\end{enumerate}	
\end{frame}

\begin{frame}{Samostatná práce – přesměrování}
	\begin{enumerate}
		\item Vytvoř soubor \texttt{vystup.log} a přesměruj do něj standardní výstup příkazu \texttt{\$ ls / (>)}.
		\item Spusť neexistující příkaz, např. \texttt{\$ ls /neexistuje}, \\ a přesměruj chybový výstup do souboru \texttt{chyby.log (2>)}.
		\item Spoj výstup a chyby do jednoho souboru \texttt{vse.log (>, 2>\&1)}.
		\item Spusť příkaz: \texttt{\$ ls /etc /neexistuje} , \\ a přesměruj jeho standardní výstup do \texttt{/dev/null}. Co se zobrazilo?
	\end{enumerate}
\end{frame}
	
\section{Signály}
	
\begin{frame}[fragile]{Co je signál}
	\begin{itemize}
		\item Signál je krátká zpráva
		\item Slouží k upozornění procesu, že se něco stalo (např. uživatel stiskl Ctrl-C). 
		\vspace{1em}
    		\item Typické signály: 
    \begin{itemize}
		\item \texttt{SIGINT} – přerušení (Ctrl + C)
		\item \texttt{SIGTSTP} – zastavení (Ctrl + Z)
		\item \texttt{SIGTERM} – žádost o ukončení
		\item \texttt{SIGKILL} – okamžité ukončení
	\end{itemize}
	\end{itemize}
	\vspace{3em}
	\textit{Pozn.: Seznam všech signálů najdete v} \texttt{\$ man 7 signal}.

\end{frame}

\begin{frame}[fragile]{Zasílání signálů}
	\begin{itemize}
		\item Signály se posílají příkazem \texttt{kill}.
		\item Syntaxe: \texttt{kill -SIGNÁL <PID>}
	\end{itemize}
	
	\vspace{0.5em}
	\hspace{2em}\texttt{\$ sleep 100 \&} \\
	\hspace{2em}[1] 5423 \\
	\hspace{2em}\texttt{\$ kill -SIGSTOP 5423} \\[0.5em]
	
	\begin{itemize}

		\item Seznam všech signálů: \texttt{kill -l}
	\end{itemize}
\end{frame}
	
\begin{frame}[fragile]{Samostatná práce – práce se signály}
	
	\begin{enumerate}
		\item Spusť: \texttt{\$ sleep 1000 \&} 
		\item Podívej se, že proces běží na pozadí. (jobs)
		\item Zastav běžící proces s pomocí \texttt{\$ kill -SIGSTOP PID}.
		\item Zkontroluj, že je proces zastavený. (jobs)
		\item Spusť proces pomocí \texttt{\$ kill -SIGCONT PID}.
		\item Zkontroluj, že proces běží. (jobs, ps)
		\item Ukonči proces pomocí \texttt{\$ kill -SIGKILL PID}.
	\end{enumerate}			
\end{frame}

\end{document}