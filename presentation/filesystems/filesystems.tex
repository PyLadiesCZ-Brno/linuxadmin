\documentclass{beamer}
\usepackage[utf8]{inputenc}
\usepackage[T1]{fontenc}
\usepackage[czech]{babel}

\setbeamerfont{title}{size=\LARGE}

\title{Souborové systémy}
\author{Eliška Jégrová}
\date{24. 11. 2025}

% pro obrázky a kreslení
\usepackage{tikz}
\usetikzlibrary{positioning}
\usepackage{graphicx}
\usepackage{datetime}
\usepackage{svg}
\usepackage{listings}
\usepackage{dirtree}


% vlastní příkaz pro tildy
\newcommand{\ts}{\raisebox{-0.25em}{\textasciitilde}}

\begin{document}
	
	\frame{\titlepage}
	
	\begin{frame}{Obsah}
		\tableofcontents
	\end{frame}


\section{Adresářová struktura}

\begin{frame}{Kořenový adresář /}
	\begin{itemize}
		\item V Linuxu jsou všechny soubory v jednom stromu pod \texttt{/}.
		\item Na rozdíl od Windows (disky \texttt{C:}, \texttt{D:}, …)
		je tu \textbf{jeden společný strom}.
		\item Strukturu \texttt{/} popisuje standard
		\textbf{Filesystem Hierarchy Standard (FHS)}.
		\item Cíle:
		\begin{itemize}
			\item oddělit systémové soubory od dat uživatelů,
			\item rozlišit soubory jen ke čtení vs. pro zápis,
			\item umožnit sdílení částí systému mezi více počítači.
		\end{itemize}
	\end{itemize}
\end{frame}

\begin{frame}{Adresářový strom – základní přehled}
	\begin{columns}[t]
		\column{0.45\textwidth}
		\dirtree{%
			.1 /.
			.2 bin.
			.2 boot.
			.2 dev.
			.2 etc.
			.2 home.
			.2 root.
			.2 run.
			.2 opt.
			.2 tmp.
			.2 usr.
			.2 var.
		}
		
		\column{0.55\textwidth}
		\begin{itemize}
		\item Tenhle základ najdeš \\(s drobnými rozdíly) na všech unixových systémech.
        \item Uživatelé pracují nejčastěji \\s \texttt{/home},
případně s \texttt{/etc} (konfigurace) a logy \\ve \texttt{/var/log}.
		\end{itemize}
	\end{columns}
\end{frame}



\begin{frame}{/bin, /usr/bin – programy}
	\begin{itemize}
		\item \texttt{/bin} – základní příkazy systému
		(např. \texttt{bash}, \texttt{ls}, \texttt{cat}).
		\item \texttt{/usr/bin} – „většina ostatních“ programů.
		\item Na moderní Fedoře je \texttt{/bin} jen odkaz na \texttt{/usr/bin}.
		\item Historicky bylo rozdělení důležité kvůli sdílení po síti;
		dnes zůstalo hlavně kvůli kompatibilitě.
	\end{itemize}
\end{frame}

\begin{frame}{/dev a /boot}
	\begin{itemize}
		\item \texttt{/dev} – speciální soubory reprezentující zařízení:
		\begin{itemize}
			\item např. \texttt{/dev/sda}, \texttt{/dev/tty0}, \texttt{/dev/null}.
			\item platí „všechno je soubor“ – čtení/zápis se mapuje na zařízení.
		\end{itemize}
		\item \texttt{/boot} – soubory potřebné ke startu systému:
		\begin{itemize}
			\item linuxové jádro,
			\item konfigurace zavaděče (bootloaderu),
			\item případně více verzí jadra pro návrat „zpět“.
		\end{itemize}
	\end{itemize}
\end{frame}

\begin{frame}{/etc, /home, /root}
	\begin{itemize}
		\item \texttt{/etc} – konfigurace systému:
		\begin{itemize}
			\item např. \texttt{/etc/passwd}, \texttt{/etc/fstab}, \texttt{/etc/ssh/sshd\_config}.
			\item většinou textové soubory, upravuje je jen \textbf{root}.
		\end{itemize}
		
		\item \texttt{/home} – domovské adresáře běžných uživatelů:
		\begin{itemize}
			\item \texttt{/home/uzivatel}, zkráceně \texttt{\ts}.
			\item sem si ukládají vlastní data a konfiguraci.
		\end{itemize}
		
		\item \texttt{/root} – domovský adresář uživatele \textbf{root}:
		\begin{itemize}
			\item root nemá home v \texttt{/home}, ale právě v \texttt{/root},
			\item běžný uživatel sem normálně nemá přístup.
		\end{itemize}
	\end{itemize}
\end{frame}

\begin{frame}{/var, /tmp, /run, /opt}
	\begin{itemize}
		\item \texttt{/var} – „proměnná“ data systému:
		\begin{itemize}
			\item logy (\texttt{/var/log}),
			\item maily, spool tiskáren,
			\item databáze a pracovní data služeb.
		\end{itemize}
		
		\item \texttt{/tmp} – dočasné soubory:
		\begin{itemize}
			\item používají ho aplikace jako odkladiště,
			\item obsah se často maže po restartu.
		\end{itemize}
		
		\item \texttt{/run} – běhová data po startu systému:
		\begin{itemize}
			\item PID soubory, sockety, různé „stavové“ informace,
			\item existují jen za běhu (po rebootu se vytváří znovu).
		\end{itemize}
		
		\item \texttt{/opt} – „volitelné“ aplikace:
		\begin{itemize}
			\item typicky software třetích stran,
			\item může mít vlastní podadresář, např. \texttt{/opt/firma/app}.
		\end{itemize}
	\end{itemize}
\end{frame}


\section{Souborové systémy}

\begin{frame}{Co je souborový systém}
	\begin{itemize}
		\item Na disku jsou data jako dlouhá řada jedniček a nul.
		\item \textbf{Souborový systém} je dohoda, jak z toho udělat:
		\begin{itemize}
			\item soubory se jménem,
			\item adresáře (strom),
			\item informace o velikosti, vlastnících, právech, časech změny, …
		\end{itemize}
		\item Umožní:
		\begin{itemize}
			\item rychle najít soubor podle cesty,
			\item přidávat/mazat soubory,
			\item kontrolovat přístup pomocí práv.
		\end{itemize}
	\end{itemize}
\end{frame}

\begin{frame}{Analogie: ZIP archiv}
	\begin{itemize}
		\item ZIP soubor:
		\begin{itemize}
			\item uvnitř jsou za sebou naskládaná \textbf{data souborů},
			\item na konci je \textbf{seznam souborů} (centrální adresář ZIPu).
		\end{itemize}
		\item Seznam říká:
		\begin{itemize}
			\item jak se soubor jmenuje,
			\item kde v ZIPu data začínají,
			\item jak je soubor dlouhý.
		\end{itemize}
	\end{itemize}
	
	\begin{center}
		\includegraphics[width=0.8\textwidth]{ZIP-64_Internal_Layout}

	\end{center}
\end{frame}

\section{Inode a odkazy}

\begin{frame}{Inode – informace o souboru}
	\begin{itemize}
		\item Soubor má:
		\begin{itemize}
			\item \textbf{obsah} (data),
			\item \textbf{jméno} (cesta v adresáři),
			\item \textbf{metadata} (velikost, práva, čas změny, \dots).
		\end{itemize}
		\item V Linuxu jsou metadata uložená v \textbf{inode}:
		\begin{itemize}
			\item typ souboru (běžný, adresář, link),
			\item práva, vlastník, skupina,
			\item velikost, časy změny,
			\item počet odkazů,
			\item odkazy na datové bloky na disku.
		\end{itemize}
		\item Ke zjištění inode se používá příkaz \texttt{stat}

		\texttt{\$ stat basnicka.txt}

	\end{itemize}

\end{frame}

\begin{frame}[fragile]{Inode vs. jméno souboru}
	\begin{itemize}
		\item \textbf{Jméno} souboru je jen záznam v adresáři.
		\item Jméno ukazuje na \textbf{inode}, inode ukazuje na data.
		\item Číslo inode zjistíš např.:
		\begin{verbatim}
			$ ls -i soubor.txt
			123456 soubor.txt
			
			$ stat soubor.txt
		\end{verbatim}
		\item Když soubor \texttt{přejmenuješ}:
		\begin{verbatim}
			$ mv soubor.txt novy.txt
		\end{verbatim}
		inode zůstává stejné – změnilo se jen jméno v adresáři.
	\end{itemize}
\end{frame}
\begin{frame}[fragile]{Odkazy - Hard link}
	\begin{itemize}
		\item \textbf{Hard link} = další jméno, které ukazuje na stejný inode.
		\item Vytvoření:
		\begin{verbatim}
			$ ln puvodni.txt kopie.txt
		\end{verbatim}
		\item Oba názvy:
		\begin{itemize}
			\item mají \textbf{stejné číslo inode},
			\item ukazují na stejná data,
			\item jsou „rovnocenné“ – žádný není „ten původní“.
		\end{itemize}
		\item Když jeden z nich smažeš:
		\begin{verbatim}
			$ rm puvodni.txt
		\end{verbatim}
		data zůstanou – dokud existuje aspoň jeden hard link
		na daný inode.
	\end{itemize}
\end{frame}
\begin{frame}[fragile]{Odkazy - Symbolický (soft) link}
	\begin{itemize}
		\item \textbf{Symbolický link} (symlink, soft link):
		\begin{itemize}
			\item zvláštní soubor, který obsahuje \textbf{cestu}
			k jinému souboru,
			\item má své vlastní číslo inode.
		\end{itemize}
		\item Vytvoření:
		\begin{verbatim}
			$ ln -s cilovy_soubor link.txt
		\end{verbatim}
		\item Chování:
		\begin{itemize}
			\item když čteš \texttt{link.txt}, systém přesměruje
			přístup na cílový soubor,
			\item když se cílový soubor smaže, \texttt{link.txt}
			zůstane, ale je „rozbitý“.
		\end{itemize}
		\item Na rozdíl od hard linku může symlink:
		\begin{itemize}
			\item mířit i na jiný souborový systém,
			\item mířit na adresář.
		\end{itemize}
	\end{itemize}
\end{frame}

\begin{frame}[fragile]{Samostatná práce – inode a odkazy}
	\begin{enumerate}
		\item Vytvoř soubor \texttt{data.txt} s libovolným obsahem.
		\item Zobraz jeho inode:
		\begin{verbatim}
			$ ls -i data.txt
			$ stat data.txt
		\end{verbatim}
		\item Vytvoř hard link \texttt{data2.txt}. (ln)
		\item Zobraz si inode (krok 2) pro \texttt{data2.txt} a porovnej s předešlými výpisy.
		\item Vytvoř symlink \texttt{data-link.txt} (ln -s) a znovu se podívej na inode.
		\item Zkus smazat \texttt{data.txt} a sleduj, co se stane
		s \texttt{data2.txt} a \texttt{data-link.txt}.
	\end{enumerate}
\end{frame}

\section{Zařízení a souborové systémy}

\begin{frame}{Bloková zařízení a souborové systémy}
	\begin{itemize}
		\item Data jsou fyzicky uložená na \textbf{zařízeních}:
		\begin{itemize}
			\item pevné disky, SSD, USB flashky, virtuální disky ve VM, ISO obrazy…
		\end{itemize}
		\item V Linuxu jsou vidět jako \textbf{bloková zařízení} v \texttt{/dev}:
		\begin{itemize}
			\item např. \texttt{/dev/sda}, \texttt{/dev/sda1}, \texttt{/dev/vda}, \dots
		\end{itemize}
		\item Na zařízení může být:
		\begin{itemize}
			\item přímo souborový systém (např. \texttt{/dev/sda}),
			\item nebo tabulka oddílů a souborový systém až na oddílu
			(např. \texttt{/dev/sda1}).
		\end{itemize}
		\item \textbf{Souborový systém} (ext4, xfs, vfat, \dots) určuje,
		jak jsou data a adresáře na zařízení uspořádané.
	\end{itemize}
\end{frame}

\begin{frame}[fragile]{lsblk – přehled disků a oddílů}
	\begin{itemize}
		\item Příkaz \texttt{lsblk} zobrazí bloková zařízení a jejich strukturu:
		\begin{verbatim}
			$ lsblk
			NAME   MAJ:MIN RM  SIZE RO TYPE MOUNTPOINTS
			vda    252:0    0   40G  0 disk
			-vda1 252:1    0    1G  0 part /boot
			-vda2 252:2    0   39G  0 part /
		\end{verbatim}
		\item Většinou uvidíš:
		\begin{itemize}
			\item disk (\texttt{TYPE disk}),
			\item na něm oddíly (\texttt{TYPE part}),
			\item u nich případně sloupec \texttt{MOUNTPOINTS}
			– kam jsou připojené.
		\end{itemize}
		\item Hodí se před každou prací s mountem – ať víš, jak se zařízení jmenují.
	\end{itemize}
\end{frame}

\begin{frame}[fragile]{df – využití souborových systémů}
	\begin{itemize}
		\item \texttt{df} ukazuje, jak jsou souborové systémy velké a zaplněné:
		\begin{verbatim}
			$ df -h
			Filesystem  Size  Used Avail Use% Mounted on
			/dev/vda2    39G   15G   22G  41% /
			tmpfs       2.0G     0  2.0G   0% /tmp
		\end{verbatim}
		\item Užitečné sloupce:
		\begin{itemize}
			\item \texttt{Filesystem} – zařízení nebo název FS,
			\item \texttt{Size}, \texttt{Used}, \texttt{Avail} – velikost a využití,
			\item \texttt{Mounted on} – kam je FS připojený.
		\end{itemize}
		\item Přepínač \texttt{-h} (human-readable) ukazuje velikosti
		v MB/GB místo v blocích.
	\end{itemize}
\end{frame}


\section{Připojování souborových systémů}

\begin{frame}[fragile]{findmnt / – co je připojeno jako /}
	\begin{itemize}
		\item Příkaz \texttt{findmnt} ukáže, jaký souborový systém je
		připojený na daném adresáři.
		\item Začneme kořenovým adresářem:
		\begin{verbatim}
			$ findmnt /
			TARGET SOURCE    FSTYPE OPTIONS
			/      /dev/sda2 ext4   rw,relatime
		\end{verbatim}
		\item Sloupce:
		\begin{itemize}
			\item \texttt{TARGET} – kam je FS připojený (adresář),
			\item \texttt{SOURCE} – odkud se bere (např. zařízení nebo jméno),
			\item \texttt{FSTYPE} – typ souborového systému,
			\item \texttt{OPTIONS} – základní připojovací volby.
		\end{itemize}
	\end{itemize}
\end{frame}

\begin{frame}[fragile]{findmnt – další příklady}
	\begin{itemize}
		\item Stejným způsobem se můžeme podívat na další mountpointy:
		\begin{verbatim}
			$ findmnt /boot
			TARGET SOURCE    FSTYPE OPTIONS
			/boot  /dev/sda1 ext4   rw,relatime
		\end{verbatim}
		
		\begin{verbatim}
			$ findmnt /home
			TARGET SOURCE      FSTYPE OPTIONS
			/home  /dev/sda3   ext4   rw,relatime
		\end{verbatim}
		\item Dobré otázky pro studenty:
		\begin{itemize}
			\item Který souborový systém je připojený na \texttt{/boot}?
			\item Který na \texttt{/home}?
		\end{itemize}
	\end{itemize}
\end{frame}

\begin{frame}[fragile]{findmnt – přehled připojených souborových systémů}
	\begin{itemize}
		\item \texttt{findmnt} ukazuje, co je kde připojeno, v podobě stromu:
		\begin{verbatim}
			$ findmnt
			TARGET  SOURCE     FSTYPE OPTIONS
			/       /dev/vda2  ext4   rw,relatime
			|-/boot /dev/vda1  ext4   rw,relatime
			`-/tmp  tmpfs      tmpfs  rw,nosuid,nodev
		\end{verbatim}
		\item Přehlednější než čistý výstup \texttt{mount}.
		\item Hodí se pro ověření:
		\begin{itemize}
			\item jestli se souborový systém připojil tak, jak jsme chtěli,
			\item co všechno je připojené pod \texttt{/mnt}, \texttt{/media}, \dots
		\end{itemize}
	\end{itemize}
\end{frame}


\begin{frame}{Mount point – princip}
	\begin{itemize}
		\item V Linuxu je jeden společný strom adresářů pod \texttt{/}.
		\item \textbf{Mount point} je adresář, do kterého se připojí
		souborový systém.
		\item Příklady:
		\begin{itemize}
			\item \texttt{/} – hlavní souborový systém,
			\item \texttt{/boot} – zvlášť připojený souborový systém,
			\item dočasně můžeme něco připojit např. do \texttt{/mnt}.
		\end{itemize}
		\item Příkaz \texttt{findmnt} nám pomáhá zjistit, co je kam připojeno.
	\end{itemize}
\end{frame}

\begin{frame}[fragile]{mount – ruční připojení}
	\begin{itemize}
		\item Ruční připojení souborového systému:
		\begin{verbatim}
			# mount zdroj cílový_adresář
		\end{verbatim}
		\item Typický scénář v návodu:
		\begin{itemize}
			\item vytvoří se adresář pro připojení, např. \texttt{/mnt/test},
			\begin{verbatim}
				# mkdir /mnt/test
			\end{verbatim}
			\item použije se \texttt{mount}, aby byl obsah vidět v \texttt{/mnt/test}.
		\end{itemize}
		\item Po připojení můžeme zkontrolovat:
		\begin{verbatim}
			$ findmnt /mnt/test
		\end{verbatim}
	\end{itemize}
\end{frame}

\begin{frame}{Mount point – připojení souborového systému}
	\begin{itemize}
		\item V Linuxu je jeden společný strom adresářů pod \texttt{/}.
		\item \textbf{Mount point} = adresář, do kterého „vložíme“ jiný souborový systém.
		\item Příklad:
		\begin{itemize}
			\item \texttt{/} – hlavní (kořenový) souborový systém,
			\item \texttt{/boot} – může být jiný FS na jiném oddílu,
			\item USB flashka může být připojená třeba do \texttt{/run/media/...}
			nebo \texttt{/mnt/usb}.
		\end{itemize}
		\item Bez mountu zařízení existuje, ale není nikde vidět ve stromu souborů.
	\end{itemize}
\end{frame}

\begin{frame}[fragile]{mount – ruční připojení}
	\begin{itemize}
		\item Základní tvar:
		\begin{verbatim}
			# mount zařízení adresář
		\end{verbatim}
		\item Příklad (USB disk s jedním oddílem):
		\begin{verbatim}
			# mkdir /mnt/usb
			# mount /dev/sdb1 /mnt/usb
		\end{verbatim}
		\item Po připojení uvidíš obsah flashky v \texttt{/mnt/usb}.
		\item Typ souborového systému (\texttt{-t ext4}, \texttt{-t vfat}, \dots)
		se většinou pozná automaticky.
		\item Co je kde připojeno, zjistíš např.:
		\begin{verbatim}
			$ findmnt
		\end{verbatim}
	\end{itemize}
\end{frame}


\begin{frame}[fragile]{umount – odpojení}
	\begin{itemize}
		\item Odpojení souborového systému:
		\begin{verbatim}
			# umount /mnt/test
		\end{verbatim}
		\item Lze odpojit i podle zdroje:
		\begin{verbatim}
			# umount zdroj
		\end{verbatim}
		\item Pokud je souborový systém používán (otevřený soubor,
		shell v tom adresáři), \texttt{umount} selže.
		\item Po odpojení \texttt{findmnt /mnt/test} nic nenajde.
	\end{itemize}
\end{frame}


\begin{frame}[fragile]{umount – odpojení souborového systému}
	\begin{itemize}
		\item Odpojení:
		\begin{verbatim}
			# umount /mnt/usb
		\end{verbatim}
		\item nebo podle zařízení:
		\begin{verbatim}
			# umount /dev/sdb1
		\end{verbatim}
		\item Když je souborový systém „zaneprázdněný“:
		\begin{itemize}
			\item někdo má otevřený soubor,
			\item máš otevřený shell v tom adresáři,
			\item bude hláška typu „target is busy“.
		\end{itemize}
		\item Řešení:
		\begin{itemize}
			\item přejít jinam (\texttt{cd}) a zavřít programy,
			\item pak zkusit \texttt{umount} znovu.
		\end{itemize}
	\end{itemize}
\end{frame}


\begin{frame}[fragile]{/etc/fstab – připojení při startu}
	\begin{itemize}
		\item Soubor \texttt{/etc/fstab} říká, které souborové systémy
		se připojí automaticky při startu.
		\item Ukázka jednoduché konfigurace:
		\begin{verbatim}
			/dev/sda2  /      ext4  defaults  1 1
			/dev/sda1  /boot  ext4  defaults  1 2
		\end{verbatim}
		\item Základní sloupce:
		\begin{itemize}
			\item \textbf{zdroj} – napr. \texttt{/dev/sda2} nebo UUID,
			\item \textbf{cílový adresář} – mount point,
			\item \textbf{typ} – typ souborového systému,
			\item \textbf{options} – základní volby (\texttt{defaults} apod.).
		\end{itemize}
	\end{itemize}
\end{frame}

\begin{frame}[fragile]{/etc/fstab – co se má připojit při startu}
	\begin{itemize}
		\item Soubor \texttt{/etc/fstab} říká, co a kam se má připojit
		při startu systému.
		\item Každý řádek popisuje jeden souborový systém:
		\begin{verbatim}
			/dev/vda2  /      ext4  defaults  1 1
			/dev/vda1  /boot  ext4  defaults  1 2
		\end{verbatim}
		\item Důležité sloupce:
		\begin{itemize}
			\item \textbf{co} – zařízení nebo UUID,
			\item \textbf{kde} – mount point,
			\item \textbf{typ} – např. \texttt{ext4}, \texttt{xfs}, \texttt{vfat}, \dots
			\item \textbf{options} – nastavení připojení (pro začátek stačí \texttt{defaults}).
		\end{itemize}
		\item Nemusíš znát všechny detaily, ale je dobré vědět,
		že automatické mounty se řídí právě tímto souborem.
	\end{itemize}
\end{frame}

\begin{frame}[fragile]{Samostatná práce – findmnt, mount, umount}
	\begin{enumerate}
		\item Pomocí \texttt{findmnt /} zjisti, jaký souborový systém je
		připojený na kořenový adresář.
		\item Zkus \texttt{findmnt /boot} a \texttt{findmnt /home}
		(pokud existuje) a podívej se, jak se liší \texttt{SOURCE}
		a \texttt{FSTYPE}.
		\item Vytvoř adresář \texttt{/mnt/test} a podle návodu na cvičení
		do něj připoj souborový systém. Ověř:
		\begin{verbatim}
			$ findmnt /mnt/test
		\end{verbatim}
		\item Souborový systém opatrně odpoj pomocí \texttt{umount}
		a zkontroluj, že \texttt{findmnt /mnt/test} už nic nevrací.
	\end{enumerate}
\end{frame}

\begin{frame}[fragile]{Samostatná práce – mount a přehled FS}
	\begin{enumerate}
		\item Pomocí \texttt{lsblk} zjisti, jaká bloková zařízení a oddíly
		má tvoje virtuálka.
		\item Zobraz přehled souborových systémů a jejich využití:
		\begin{verbatim}
			$ df -h
		\end{verbatim}
		\item Podívej se, co je právě připojeno:
		\begin{verbatim}
			$ findmnt
		\end{verbatim}
		\item Otevři \texttt{/etc/fstab} (např. \texttt{less /etc/fstab})
		a najdi, který řádek odpovídá kořenovému souborovému systému \texttt{/}
		a který \texttt{/boot}.
		\item Pokud máš v prostředí připravený další disk/oddíl,
		zkus ho ručně připojit do \texttt{/mnt/test}
		a ověř, že se objeví ve výstupu \texttt{findmnt}.
	\end{enumerate}
\end{frame}

\end{document}